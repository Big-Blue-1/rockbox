% $Id$ %
\chapter{\label{ref:rockbox_interface}Quick Start}
\section{Basic Overview}
\subsection{The \daps{} controls}

% include the front image. Using \specimg makes this fairly easy,
% but requires to use the exact value of \specimg in the filename!
% The extension is selected in the preamble, so no further \Ifpdfoutput
% is necessary.
%
% The check looks for a png file -- we use png for the HTML manual, so that
% format needs to be present. It can also be used for the pdf manual, but
% usually we provide a pdf version of the file for that. Picking the correct
% one is done by LaTeX automatically, but for checking the filename we need to
% specify the extension.
\begin{center}
\IfFileExists{rockbox_interface/images/\specimg-front.png}
   {\Ifpdfoutput{\includegraphics[height=8cm,width=10cm,keepaspectratio=true]%
    {rockbox_interface/images/\specimg-front}}
   {\includegraphics{rockbox_interface/images/\specimg-front}}
   }
   {\color{red}{\textbf{WARNING!} Image not found}%
    \typeout{Warning: missing front image}
   }
\end{center}
\opt{HAVEREMOTEKEYMAP}{
  % spacing between the two pictures, could possibly be improved
  \begin{center}
  \IfFileExists{rockbox_interface/images/\specimg-remote.png}
    {\Ifpdfoutput{\includegraphics[height=5.6cm,width=10cm,keepaspectratio=true]{rockbox_interface/images/\specimg-remote}}
    {\includegraphics{rockbox_interface/images/\specimg-remote}}
    }
    {\color{red}{\textbf{WARNING!} Image not found}%
     \typeout{Warning: missing remote image}
    }
  \end{center}
}

Throughout this manual, the buttons on the \dap{} are labelled according to the
picture above.

\opt{erosqnative}{
  \note{The \dap{} has two outputs: Headphone output on the left and Line output on the right.
  Due to the circuitry which does detection, it is recommended to only use headphones plugged
  into the Headphone output and devices which take Line input into the Line output. If devices
  which take a Line input are plugged into the Headphone output, it is likely that the \dap{}
  will not detect the plug and no sound will result.

  In the other direction, headphones plugged into the Line output may damage the headphones (and
  your ears!) due to the extremely loud volume.

  Note that the volume of the Line output is set by the \setting{Volume Limit} if it needs to
  be reduced.}
}

\opt{touchscreen}{
The areas of the touchscreen in the 3$\times$3 grid mode are in turn referred as follows:
\begin{table}
    \centering
    \begin{tabular}{|c|c|c|}
	\hline
        \TouchTopLeft & \TouchTopMiddle & \TouchTopRight \\ [5ex]
	\hline
	\TouchMidLeft & \TouchCenter & \TouchMidRight \\ [5ex]
	\hline
	\TouchBottomLeft & \TouchBottomMiddle & \TouchBottomRight \\ [5ex]
	\hline
    \end{tabular}
\end{table}
}%
Whenever a button name is prefixed by ``Long'', a long press of approximately
one second should be performed on that button.

The buttons are described in
detail in the following paragraph.

\blind{%
  Additional information for blind users is available on the Rockbox website at
  \wikilink{BlindFAQ}.

  %
  \opt{iriverh100}{
  Hold or lay the \dap{} so that the side with the joystick and LCD is facing
  towards you, and the curved side is at the top. The joystick functions as
  the \ButtonUp{}, \ButtonRight{}, \ButtonLeft{}, and \ButtonDown{} buttons when
  pressed in the appropriate direction. Pressing the joystick down functions as
  \ButtonSelect{}.
  On the right side of the \dap{} are the \ButtonOn{}, \ButtonOff{},
  \ButtonMode{} buttons, and the \ButtonHold{} switch. When this switch is
  switched towards the bottom of the \dap{}, hold is on, and none of the other
  buttons have any effect.

  On the left side is the \ButtonRec{} button. Above that is the internal microphone.

  On the top panel of the \dap{}, from left to right, you can find the
  following: headphone mini jack plug, remote port, Optical line-in, Optical line-out.

  On the bottom panel of the \dap{}, from left to right, you can find the
  following: power jack, reset switch, and USB port. In the event that your
  \dap{} hard locks, you can reset it by inserting a paper clip into the hole
  where the reset switch is.}
  %
  \opt{iriverh300}{
  Hold or lay the \dap{} so that the side with the button pad and
  LCD is facing towards you.  The buttons on the button pad are as follows:  top
  left corner: \ButtonOn{}, bottom left corner: \ButtonOff{}, top right corner:
  \ButtonRec, bottom right corner: \ButtonMode{}.  In the center of the button pad
  is a button labelled \ButtonSelect{}.  Surrounding the \ButtonSelect{} button are
  the \ButtonUp{}, \ButtonDown{}, \ButtonLeft{}, and \ButtonRight{} buttons.

  On the top panel of the \dap{}, from left to right, you can find the
  following: headphone mini jack plug, remote port, line-in, line-out.

  On the left hand side of the \dap{} is the internal microphone. Just underneath
  this is a small hole, the reset switch. In the event that your \dap{} hard locks,
  you can reset it by inserting a paper clip into the hole where the reset switch
  is.

  On the right hand side of the \dap{} is the \ButtonHold{} switch. When this is
  switched towards the bottom of the \dap{}, hold is on, and none of the other
  buttons have any effect.

  On the bottom panel of the \dap{}, from left to right, you can find the
  following:  power jack and two USB ports.  The USB port on the right is used
  to connect your \dap{} to your computer.  The USB port on the left is not
  used in Rockbox.
  }
  %
  \opt{mpiohd200}{
  Hold or lay the \dap{} so that the side with the LCD is facing towards you.
  On the right hand side there is a rocker switch at the top which serves as
  \ButtonRew{} and \ButtonFF{} when rocked up or down, respectively.
  Pressing the rocker in functions as the \ButtonFunc{} button. Below the rocker
  there are the \ButtonRec{} and \ButtonPlay{} buttons. At the bottom of the
  right panel there is the \ButtonHold{} switch. When this is switched towards the
  bottom of the \dap{}. hold is on, and none of the other buttons have any effect.

  On the top panel of the \dap{} there is another rocker which serves as the
  \ButtonVolDown{} and \ButtonVolUp{} buttons when pressed to the left or right,
  respectively.

  On the left hand side of the \dap{} there is a headphone mini jack plug at the top
  and a small hole at the bottom, the reset switch. In the event that your \dap{}
  hard locks, you can reset it by inserting a paper clip into the hole where the
  reset switch is.

  On the bottom panel of the \dap{}, from left to right, you can find the
  following: power jack, line-in jack and USB port (under rubber cover).
  }
  %
  \opt{ipod4g,ipodcolor,ipodvideo,ipodmini}{
  The main controls on the \dap{} are a slightly indented scroll wheel
  with a flat round button in the center. Hold the \dap{} with these controls
  facing you.

  The top of the player will have the following, from left to
  right:
  \opt{ipod4g,ipodcolor}{remote connector, headphone socket, \ButtonHold{}
    switch.}
  \opt{ipodvideo}{\ButtonHold{} switch, headphone socket.}
  \opt{ipodmini}{\ButtonHold{} switch, remote connector, headphone socket.}

  The dock connector that is used to connect your \dap{} to your computer is on
  the bottom panel of the \dap{}.

  The button in the middle of the wheel is called \ButtonSelect{}. You can
  operate the wheel by pressing the top, bottom, left or right sections,
  or by sliding your finger around it.  The top is \ButtonMenu{}, the bottom is
  \ButtonPlay{}, the left is \ButtonLeft{}, and the right is \ButtonRight{}.
  When the manual says to \ButtonScrollFwd{}, it means to slide your finger
  clockwise around the wheel. \ButtonScrollBack{} means to slide your finger
  counterclockwise. Note that the wheel is sensitive, so you will need to move
  slowly at first and get a feel for how it works.

  Note that when the \ButtonHold{} switch is pushed toward the center of the \dap{},
  hold is on, and none of the other controls do anything.  Be sure
  \ButtonHold{} is off before trying to use your player.
  }
  %
  \opt{ipod3g}{
  The main controls on the \dap{} are a slightly indented touch wheel
  with a flat round button in the center, and four buttons in a row above the
  touch wheel. Hold the \dap{} with these controls
  facing you.

  The top of the player will have the following, from left to
  right: remote connector, headphone socket, \ButtonHold{} switch.

  The dock connector that is used to connect your \dap{} to your computer is on
  the bottom panel of the \dap{}.

  The button in the middle of the wheel is called \ButtonSelect{}. You can
  operate the wheel by sliding your finger around it.  The row of
  buttons consists of, from left to right, the \ButtonLeft{},
  \ButtonMenu{}, \ButtonPlay{}, and \ButtonRight{} buttons.
  When the manual says to \ButtonScrollFwd{}, it means to slide your finger
  clockwise around the wheel. \ButtonScrollBack{} means to slide your finger
  counterclockwise. Note that the wheel is sensitive, so you will need to move
  slowly at first and get a feel for how it works.

  Note that when the \ButtonHold{} switch is pushed toward the center of the \dap{},
  hold is on, and none of the other controls do anything.  Be sure
  \ButtonHold{} is off before trying to use your player.
  }
  %
  \opt{ipod1g2g}{
  The main controls on the \dap{} are a slightly indented wheel
  with a flat round button in the center, and four buttons surrounding
  it. On the 1st generation iPod, this wheel physically turns. On the
  2nd generation iPod, this wheel is touch-sensitive. Hold the \dap{} with these controls
  facing you.

  The top of the player will have the following, from left to
  right: FireWire port, headphone socket, \ButtonHold{} switch.

  The FireWire port is used to connect your \dap{} to the computer and
  to charge its battery via a wall charger.

  The button in the middle of the wheel is called \ButtonSelect{}. You can
  operate the wheel by turning it, or sliding your finger around
  it. The top is \ButtonMenu{}, the bottom is \ButtonPlay{}, the left
  is \ButtonLeft{}, and the right is \ButtonRight{}.
  When the manual says to \ButtonScrollFwd{}, it means to slide your finger
  clockwise around the wheel. \ButtonScrollBack{} means to slide your finger
  counterclockwise. Note that the wheel is sensitive, so you will need to move
  slowly at first and get a feel for how it works.

  Note that when the \ButtonHold{} switch is pushed toward the center of the \dap{},
  hold is on, and none of the other controls do anything.  Be sure
  \ButtonHold{} is off before trying to use your player.
  }
  %
  \opt{ipodnano,ipodnano2g}{
  The main controls on the \dap{} are a slightly indented wheel with a
  flat round button in the center. Hold the \dap{} with these controls on the
  top surface. There is a \ButtonHold{} switch at one end, and
  headphone and dock connector at the other; be sure the end with the
  switch is facing away from you.

  The button in the middle of the wheel is called \ButtonSelect{}. You can
  operate the wheel by pressing the top, bottom, left or right sections,
  or by sliding your finger around it.  The top is \ButtonMenu{}, the bottom is
  \ButtonPlay{}, the left is \ButtonLeft{}, and the right is \ButtonRight{}.
  When the manual says to \ButtonScrollFwd{}, it means to slide your finger
  clockwise around the wheel. \ButtonScrollBack{} means to slide your finger
  counterclockwise. Note that the wheel is sensitive, so you will need to move
  slowly at first and get a feel for how it works.

  Note that when the \ButtonHold{} switch is pushed toward the center of the \dap{},
  hold is on, and none of the other controls do anything; be sure \ButtonHold{} is
  off before trying to use your player.
  }
  %
  \opt{iriverh10,iriverh10_5gb}{
  Hold or lay the \dap{} so that the side with the scroll pad and
  LCD is facing towards you. In the centre below the lcd is the scroll pad. It
  is oriented vertically. Touching the top and bottom half of it acts as the
  \ButtonScrollUp{}  and \ButtonScrollDown{} buttons respectively. On the left
  of the scroll pad is the \ButtonLeft{} button and on the right is the
  \ButtonRight{} button.

  There are three buttons on the right hand side of the \dap{}. From top to
  bottom, they are: \ButtonRew{}, \ButtonPlay{} and \ButtonFF{}. On the left
  hand side is the \ButtonPower{} button.

  On the top panel of the \dap{}, from left to right, you can find the
  following: \ButtonHold{} switch, \opt{iriverh10}{reset pin hole, }remote port
  and headphone mini jack plug.

  On the bottom panel of the \dap{} is the data cable port.}
  %
  \opt{gigabeatf}{
  \note{The following description is for the Gigabeat F, but can also apply for the
  Gigabeat X. The Gigabeat F is slightly larger and more rectangular shaped, while the
  Gigabeat X is smaller and has a slightly tapered back.}

  Hold the \dap{} with the screen on top and the controls on the right hand side.
  Below the screen is a cross-shaped touch sensitive pad which contains the
  \ButtonUp{}, \ButtonDown{}, \ButtonLeft{} and \ButtonRight{} controls.  On the
  Gigabeat X, this pad will feel slightly raised up, while it will feel slightly
  sunken in on the Gigabeat F. On the top of the unit, from left to right, are the
  power socket, the \ButtonHold{} switch, and the headphone socket.  The
  \ButtonHold{} switch puts the \dap{} into hold mode when it is switched to the
  right of the unit. The buttons will have no effect when this is the case.

  Starting from the left hand side on the bottom of the unit, nearer to the front
  than the back, is a recessed switch which
  controls whether the battery is on or off.  When this switch is to the left,
  the battery is disconnected.  This can be used for a hard reset of the unit,
  or if the \dap{} is being placed in storage.  Next to that is a connector for
  the docking station and finally on the right hand side of the bottom of the
  unit is a mini USB socket for connecting directly to USB.

  Finally on the right hand side of the unit are some control buttons.  Going from
  the bottom of the unit to the top there is a small round \ButtonA{} buttton then a
  rocker volume switch with of the \ButtonVolDown{} button below the \ButtonVolUp{}
  button.  Above that is are two more small round buttons, the \ButtonMenu{}
  button and nearest to the top of the unit the \ButtonPower{} button, which is held
  down to turn the \dap{} on or off. If you have a Gigabeat X, these buttons are small
  metallic buttons that are place further up on the right hand side, and closer
  together. The layout is still the same, however.}
  %
  \opt{gigabeats}{
  Hold the \dap{} with the screen on top and the controls on the right hand side.
  Directly below the bottom edge of the screen are two buttons, \ButtonBack{}
  on the left and \ButtonMenu{} on the right. Below them is a cross-shaped pad
  which contains the \ButtonUp{}, \ButtonDown{}, \ButtonLeft{}, \ButtonRight{}
  and \ButtonSelect{} controls.
  On the top of the unit from left to right are the headphone socket and the
  \ButtonHold{} switch.  The \ButtonHold{} switch puts the \dap{} into
  hold mode when it is switched to the right of the unit.
  The buttons will have no effect when this is the case.

  Starting from the left hand side on the bottom of the unit, nearer to the back
  than the front, is a recessed switch which controls whether the battery is on
  or off.  When this switch is to the left, the battery is disconnected.
  This can be used for a hard reset of the unit, or if the \dap{} is being placed
  in storage.  Next to that is a mini USB socket for connecting directly to USB,
  and finally a custom connector, presumably for planned accessories which were
  never released.

  Finally on the right hand side of the unit are some control buttons and the power
  connector.  Going from the bottom of the unit to the top, there is the power
  connector socket, followed by three small round buttons, the
  \ButtonNext{} buttton, \ButtonPlay{} button, and \ButtonPrev{} button (from bottom
  to top) then a rocker volume switch with of the \ButtonVolDown{} button below the
  \ButtonVolUp{} button.  Above that is one more small round button, the \ButtonPower{}
  button, which is held down to turn the \dap{} on or off.}
  %
  \opt{mrobe100}{
  Hold the \dap{} with the black front facing you such that the m:robe writing
  is readable. Below the writing is the touch sensitive pad with the
  \ButtonMenu{}, \ButtonPlay{}, \ButtonLeft{}, \ButtonRight{} and \ButtonDisplay
  controls indicated by their symbols. The dotted center strip is devided in
  three parts: \ButtonUp{}, \ButtonSelect{} and \ButtonDown. On the top of the
  unit, on the right, is the \ButtonPower{} switch, which is held down to turn
  the \dap{} on or off.

  The \ButtonHold{} switch is located on the left of the \dap{}, below the
  headphone socket. It puts the \dap{} into hold mode when it is switched to the
  top of the unit. The buttons will have no effect when this is the case. On the
  bottom of the unit, there is a connector for the docking station or the
  proprietary USB connector for connecting directly to USB.}
  %
  \opt{iaudiom5,iaudiox5}{
  The \dap{} is curved so that the end with the screen on it is thicker than the
  other end.  Hold the \dap{} wih the thick end towards the top and the screen
  facing towards you.  Half way up the front of the unit on the right hand side
  is a four way joystick which is the \ButtonUp{}, \ButtonDown{},
  \ButtonLeft{}, and \ButtonRight{} buttons. When pressed it serves as \ButtonSelect{}.

  On the right hand side of the \dap{} from top to bottom, first there is a two
  way switch.  the \ButtonPower{} button is activated by pushing this switch up,
  and pushing this switch down until it clicks slightly will activate the
  \ButtonHold{} button.  When the switch is in this position, none of the other
  keys will have an effect.

  Below the switch is a lozenge shaped button which is the \ButtonRec{}
  button, and below that the final button on this side of the unit, the
  \ButtonPlay{} button.  Just below this is a small hole which is difficult to
  locate by touch which is the internal microphone.  At the very bottom of
  this side of the unit is the reset hole, which can be used to perform a hard
  reset by inserting a paper clip.

  On the bottom of the unit is the connector for the
  \playerman{} subpack or dock.  On the top of the unit is a charge
  indicator light, which may feel a bit like a button, but is not.

  From the top of the \dap{} on the left hand side is the headphone socket, then the
  remote connector.  Below this is a cover which protects the \opt{iaudiox5}{USB
  host connector.}\opt{iaudiom5}{USB and charging connector}.}
  %
  \opt{e200,e200v2}{
  Hold the \dap{} with the turning wheel at the front and bottom.  On the bottom left
  of the front of the \dap{} is a raised round button, the \ButtonPower{} button.
  Above and to the left of this, on the outside of the turning wheel are four
  buttons.  These are the \ButtonUp{}, \ButtonDown{}, \ButtonLeft{} and
  \ButtonRight{} buttons.  Inside the wheel is the \ButtonSelect{} button.  Turning
  the wheel to the right activates the \ButtonScrollFwd{} function, and to the
  left, the \ButtonScrollBack{} function.

  On the right of the unit is a slot for inserting flash cards.  On the bottom is
  the connector for the USB cable.  On the left is the \ButtonRec{} button, and
  on the top, there is the headphone socket to the right, and the \ButtonHold{}
  switch.  Moving this switch to the right activates hold mode in which none of the
  other buttons have any effect.  Just to the left of the \ButtonHold{} switch is a
  small hole which contains the internal microphone.}
  %
  \opt{c200,c200v2}{
  Hold the \dap{} with the buttons on the right and the screen on the left. On
  the right side of the unit, there is a series of four connected buttons that
  form a square. The four sides of the square are the \ButtonUp{},
  \ButtonDown{}, \ButtonLeft{} and \ButtonRight{} buttons, respectively. Inside
  the square formed by these four buttons is the \ButtonSelect{} button. At the
  bottom right corner of the square is a small separate button, the
  \ButtonPower{} button.

  Moving clockwise around the outside of the unit, on the top are the \ButtonVolUp{}
  and \ButtonVolDown{} buttons, which control the volume of playback. The buttons can
  be distinguished by a sunken triangle on the \ButtonVolDown{} button, and a
  raised triangle on the \ButtonVolUp{} button. To the right of
  the volume buttons on the top of the unit is the slot for inserting flash
  memory cards. On the right side of the unit is the connector for the USB
  cable. At center of the bottom of the \dap{} is the \ButtonRec{} button. To
  the left of the \ButtonRec{} button is the \ButtonHold{} switch. Moving this
  switch to the right activates hold mode, in which none of the other buttons
  have any effect. On the lower left side of the unit is the headphone socket.
  Immediately above the headphone socket is a lanyard loop and the microphone.
  }
  %
  \opt{fuze,fuzev2}{
  Hold the \dap{} with the controls on the bottom and the screen on the top. The main
  controls are a scroll wheel with four clickable points and a button in the centre; pressing
  this centre button functions as \ButtonSelect{}. Going clockwise from the top, the clickable
  points on the wheel are the \ButtonUp{}, \ButtonRight{}, \ButtonDown{}, and \ButtonLeft{}
  buttons. Turning the wheel clockwise is \ButtonScrollFwd{}, and turning it counter-clockwise
  is \ButtonScrollBack{}. Immediately above and to the right of the wheel is the \ButtonHome{}
  button.

  On the lower left of the unit is a slot for inserting microSD cards. Immediately below that is
  the opening for the microphone.

  On the bottom of the unit is the connector for connecting a USB cable and the headphone socket.
  On the lower right hand side of the unit is a two-way switch. Pressing this switch up acts as
  \ButtonPower{}, and clicking it down until it locks acts as the \ButtonHold{} switch. When the
  \ButtonHold{} switch is on, none of the other buttons have any effect.
  }
  %
  \opt{clipplus,clipv1,clipv2,clipzip}{
  Hold the \dap{} with the controls on the bottom and the screen on the top. The main
  controls are a four-way pad with a button in the centre; pressing this centre button
  functions as \ButtonSelect{}. Going clockwise from the top, the four-way pad contains
  the \ButtonUp{}, \ButtonRight{}, \ButtonDown{}, and \ButtonLeft{} buttons.
  Immediately above and to the \nopt{clipzip}{right}\opt{clipzip}{left} of the four-way
  pad is the \ButtonHome{} button.
  }
  %
  \opt{clipplus,clipzip}{
  The \ButtonPower{} button is on the top of the \dap{}\opt{clipplus}{, towards the right side.}

  At the bottom of the right side of the \dap{} is a slot for microSD cards.
  Above this slot on the right side is the headphone socket.

  On the left hand panel is a two-way button that acts as \ButtonVolDown{} when
  pressed on the bottom, and \ButtonVolUp{} when pressed on the top. Immediately
  above the switch is a mini-USB port to connect the \dap{} to a computer.

  }
  %
  \opt{clipv1,clipv2}{
  On the left hand panel is a two way switch. Pressing this switch up acts as
  \ButtonPower{}, and clicking it down until it locks acts as the \ButtonHold{}
  switch. When the \ButtonHold{} switch is on, none of the other buttons have any
  effect. Immediately above the switch is a mini-USB port to connect the \dap{} to
  a computer.

  On the right hand panel is a two-way button that acts as \ButtonVolDown{} when
  pressed on the bottom, and \ButtonVolUp{} when pressed on the top. Immediately
  above this button is the headphone socket.
  }
  %
  \opt{vibe500}{
  Hold or lay the \dap{} so that the side with the controls and
  LCD is facing towards you. Below the LCD is the touch sensitive pad with the \ButtonMenu{},
  \ButtonPlay{}, \ButtonLeft{}, \ButtonRight{} controls and the scroll pad in the centre. The
  scroll pad is oriented vertically between the \ButtonOK{} and \ButtonCancel{} buttons.
  Sliding a finger up or down the scroll pad acts as \ButtonUp{} and \ButtonDown{} respectively.
  Note that the scroll pad is sensitive, so you will need to move
  slowly at first and get a feel for how it works.

  There are two buttons on the right hand side of the \dap{}: \ButtonPower{} on the top and
  \ButtonRec{} underneath. Under these buttons, from top to bottom you can find: USB connector,
  power connector and the reset hole if you need to perform a hardware reset.

  The \ButtonHold{} switch is located on the left hand side of the \dap{}. Note that when the
  \ButtonHold{} switch is moved towards the top of the \dap{}, hold is turned on and all the
  other controls are disabled. Be sure \ButtonHold{} is off before trying to use your player.

  On the top on the \dap{} is the internal microphone on the left and the line-in socket on the
  right, near the headphone socket.}
  %
  \opt{samsungyh820}{
  Hold or lay the \dap{} so that the side with the controls and
  LCD is facing towards you. Directly below the bottom edge of the screen are three buttons:
  \ButtonRew{} on the left, \ButtonPlay{} in the middle and \ButtonFF{} on the right. Below them
  is a four-way pad which contains the \ButtonDown{}, \ButtonUp{}, \ButtonLeft{} and
  \ButtonRight{} controls.

  At the top of the right hand side of the \dap{} is the \ButtonRec{} button.

  On the top panel of the \dap{}, from left to right, you can find the following: headphone
  socket, line-in socket, internal microphone, and the \ButtonHold{} switch. Note that when the
  \ButtonHold{} switch is moved towards the center of the \dap{}, hold is turned on and all the
  other controls are disabled. Be sure \ButtonHold{} is off before trying to use your player.

  At the top of the back side of the player, just under the \ButtonHold{} button is the reset
  hole, if you need to perform a hardware reset.

  The USB/dock connector that is used to connect your \dap{} to your computer is on
  the bottom panel of the \dap{}.
  }
  %
  \opt{samsungyh920,samsungyh925}{
  Hold or lay the \dap{} so that the side with the controls and
  LCD is facing towards you. Below the LCD is a four-way pad with the \ButtonDown{},
  \ButtonUp{}, \ButtonLeft{} and \ButtonRight{} buttons.

  There are three buttons at the top of the right hand side of the \dap{}: \ButtonFF{} on the top,
  \ButtonPlay{} in the middle and \ButtonRew{} underneath. Below these buttons is the \ButtonRec{}
  switch. Rockbox doesn't take note of the actual \emph{position} of the switch, but reacts to a
  \emph{switching movement} like pressing a regular button.

  On the top panel of the \dap{}, from left to right, you can find the following: headphone/remote
  socket, line-in socket, internal microphone, and the \ButtonHold{} switch. Note that when the
  \ButtonHold{} switch is moved towards the center of the \dap{}, hold is turned on and all the
  other controls are disabled. Be sure \ButtonHold{} is off before trying to use your player.

  At the top of the back side of the player, just under the \ButtonHold{} button is the reset hole,
  if you need to perform a hardware reset.

  The USB/dock connector that is used to connect your \dap{} to your computer is on
  the bottom panel of the \dap{}.
  }
  %
  \opt{erosqnative}{
  Hold or lay the \dap{} so that the side with the controls and LCD is facing towards you.
  Below the LCD is a scrollwheel, with clockwise movement being \ButtonScrollFwd{} and counterclockwise
  movement being \ButtonScrollBack{}. In the center of the scrollwheel is \ButtonPlay{}.

  To the right of the scrollwheel are a pair of buttons, the upper one being \ButtonPrev{} (which doubles as \ButtonRW{})
  and the lower one being \ButtonNext{} (which doubles as \ButtonFF{}). Below and to the left of the scrollwheel is the
  \ButtonMenu{} button, and below and to the right of the scrollwheel is the \ButtonBack{} button.

  \note{While every effort was made to make the scrollwheel as reliable as possible, it is not perfect and sometimes
  can go two menu entries for one indent on the wheel. Fortunately, Rockbox menus (as well as original firmware menus)
  also allow using the \ButtonPrev{} and \ButtonNext{} buttons to scroll through them, which can be much more reliable.}

  On the right-hand edge of the device is, from top to bottom, \ButtonVolUp{}, \ButtonVolDown{}, and the microSD
  slot.

  On the top edge of the device is the \ButtonPower{} button.

  On the bottom edge of the device is, from left to right, the headphone jack, the USB port, and the line-out jack.
  }
}

\subsection{Turning the \dap{} on and off}
\opt{cowond2}{Rockbox has a dual-boot feature with the original firmware being
  the default.\\}
To turn on and off your Rockbox enabled \dap{} use the following keys:
    \begin{btnmap}
      \opt{IRIVER_H100_PAD,IRIVER_H300_PAD}{\ButtonOn}%
      \opt{MPIO_HD200_PAD,MPIO_HD300_PAD,SAMSUNG_YH92X_PAD,SAMSUNG_YH820_PAD}%
          {Long \ButtonPlay}%
      \opt{IPOD_4G_PAD}{\ButtonMenu{} / \ButtonSelect}%
      \opt{IPOD_3G_PAD}{\ButtonMenu{} / \ButtonPlay}%
      \opt{IAUDIO_X5_PAD,IRIVER_H10_PAD,SANSA_E200_PAD,SANSA_C200_PAD,ONDA_VX777_PAD%
          ,GIGABEAT_PAD,MROBE100_PAD,GIGABEAT_S_PAD,sansaAMS,PBELL_VIBE500_PAD%
          ,SANSA_FUZEPLUS_PAD,XDUOO_X3_PAD,AIGO_EROSQ_PAD%
          }{\ButtonPower}%
      \opt{COWON_D2_PAD} {\ButtonPower{}, then \ButtonHold}%
      \opt{ONDA_VX777_PAD} {\ButtonPower{}}%
      \opt{AGPTEK_ROCKER_PAD}{\ButtonPower{}}%
          &
      \opt{HAVEREMOTEKEYMAP}{
          \opt{IRIVER_RC_H100_PAD}{\ButtonRCOn}%
          \opt{IAUDIO_RC_PAD}{\ButtonRCPlay}
          &}

      Start Rockbox
          \\

      \opt{IRIVER_H100_PAD,IRIVER_H300_PAD}{Long \ButtonOff}%
      \opt{MPIO_HD200_PAD,MPIO_HD300_PAD,SAMSUNG_YH92X_PAD,SAMSUNG_YH820_PAD}%
          {Long \ButtonPlay}%
      \opt{IPOD_4G_PAD,IPOD_3G_PAD}{Long \ButtonPlay}%
      \opt{IAUDIO_X5_PAD,IRIVER_H10_PAD,SANSA_E200_PAD,SANSA_C200_PAD%
          ,GIGABEAT_PAD,MROBE100_PAD,GIGABEAT_S_PAD,sansaAMS,COWON_D2_PAD%
          ,PBELL_VIBE500_PAD,ONDA_VX777_PAD,SANSA_FUZEPLUS_PAD,XDUOO_X3_PAD,AIGO_EROSQ_PAD%
          }{Long \ButtonPower}%
      \opt{AGPTEK_ROCKER_PAD}{Long \ButtonPower{}}%
          &
      \opt{HAVEREMOTEKEYMAP}{
          \opt{IRIVER_RC_H100_PAD}{Long \ButtonRCStop}%
          \opt{IAUDIO_RC_PAD}{Long \ButtonRCPlay}
          &}

      Shutdown Rockbox
          \\
    \end{btnmap}

\label{ref:Safeshutdown}On shutdown, Rockbox automatically saves its settings.

\opt{IRIVER_H100_PAD,IRIVER_H300_PAD,IAUDIO_X5_PAD,SANSA_E200_PAD%
  ,SANSA_C200_PAD,IRIVER_H10_PAD,IPOD_4G_PAD,GIGABEAT_PAD}{%
  If you have problems with your settings, such as accidentally having
  set the colours to black on black, they can be reset at boot time.  See
  the Reset Settings in \reference{ref:manage_settings_menu} for details.
}%

\opt{GIGABEAT_PAD,IPOD_4G_PAD,SANSA_E200_PAD%
,SANSA_C200_PAD,IAUDIO_X5_PAD,IAUDIO_M5_PAD,IPOD_3G_PAD}{%
  In the unlikely event of a software failure, hardware poweroff or reset can be
  performed by holding down
  \opt{GIGABEAT_PAD}{the battery switch}\opt{IPOD_4G_PAD}
  {\ButtonMenu{} and \ButtonSelect{} simultaneously}%
  \opt{IPOD_3G_PAD}{\ButtonMenu{} and \ButtonPlay{} simultaneously}%
  \opt{SANSA_E200_PAD,SANSA_C200_PAD,IAUDIO_X5_PAD,IAUDIO_M5_PAD}
  {\ButtonPower} until the \dap{} shuts off or reboots.
}%
\opt{IRIVER_H100_PAD,IRIVER_H300_PAD,IAUDIO_M3_PAD,IRIVER_H10_PAD,MROBE100_PAD
    ,PBELL_VIBE500_PAD,MPIO_HD200_PAD,MPIO_HD300_PAD,SAMSUNG_YH92X_PAD%
    ,SAMSUNG_YH820_PAD,XDUOO_X3_PAD}{%
  In the unlikely event of a software failure, a hardware reset can be
  performed by inserting a paperclip gently into the Reset hole.
}%

\nopt{gigabeatf,iaudiom3,iaudiom5,iaudiox5}
  {
  \subsection{Starting the original firmware}
  \label{ref:Dualboot}
  \opt{ipod4g,ipodcolor,ipodvideo,ipodnano,ipodnano2g,ipodmini}
    {
    Rockbox has a dual-boot feature. To boot into the original firmware, shut
    down the device as described above. Turn on the \ButtonHold{} switch
    immediately after turning the player on. The Apple logo will
    display for a few seconds as Rockbox loads the original firmware.

    You can also load the original firmware by shutting down the device,
    then clicking the \ButtonHold{} switch on and connecting the iPod
    to your computer.

    Regardless of which method you use to boot to the original firmware, you can
    return to Rockbox by pressing and holding \ButtonMenu{} and \ButtonSelect{}
    simultaneously until the player hard resets.
    }

  \opt{ipod1g2g,ipod3g}
    {
    Rockbox has a dual-boot feature. To boot into the original firmware, shut
    down the device as described above. Turn on the \ButtonHold{} switch
    immediately after turning the player on. The Apple logo will
    display for a few seconds as Rockbox loads the original firmware.

    You can also load the original firmware by shutting down the device,
    then clicking the \ButtonHold{} switch on and connecting the iPod
    to your computer.

    Regardless of which method you use to boot to the original firmware, you can
    return to Rockbox by pressing and holding \ButtonMenu{} and \ButtonPlay{}
    simultaneously until the player hard resets.
    }

  \opt{iriverh100,iriverh300}
    {
    Rockbox has a dual-boot feature. To boot into the original firmware,
    when the \dap{} is turned off, press and hold the \ButtonRec{} button,
    and then press the \ButtonOn{} button.
    }
  \opt{fuzeplus}
    {
    Rockbox has a dual-boot feature. To boot into the original firmware,
    when the \dap{} is turned off, press and hold the \ButtonVolDown{} button,
    and then press and hold the \ButtonPower{} button while keeping the
    \ButtonVolDown{} button pressed. After 5 to 10 seconds the original
    firmware should boot.

    It is also possible to connect your \dap{} to your computer using the
    original firmware. To do so you may press and hold the \ButtonVolDown{}
    button and connect your device to the computer while keeping the
    \ButtonVolDown{} button pressed. After 5 to 10 seconds the original
    firmware should boot into USB mode.
    }
  \opt{mpiohd200,mpiohd300}
    {
    Rockbox has a dual-boot feature. To boot into the original firmware,
    when the \dap{} is turned off, press and hold the \ButtonRec{} button,
    and then press the \ButtonPlay{} button. This will bring you to the
    short menu where you can choose among: Boot Rockbox, Boot MPIO firmware
    and Shutdown. Select the option you need with \ButtonRew{} and \ButtonFF{}
    and confirm with long \ButtonPlay{}.
    }
  \opt{iriverh10,iriverh10_5gb}
    {
    Rockbox has a dual-boot feature. It loads the original firmware from
    the file \fname{/System/OF.mi4}. To boot into the original firmware,
    press and hold the \ButtonLeft{} button while turning on the player.
    \note{The iriver firmware does not shut down properly when you turn it off,
    it only goes to sleep. To get back into Rockbox when exiting from the
    iriver firmware, you will need to reset the player by \opt{iriverh10}{%
    inserting a pin in the reset hole}\opt{iriverh10_5gb}{removing and
    reinserting the battery}.}
    }

  \opt{sansa,sansaAMS}
    {
    Rockbox has a dual-boot feature. To boot into the original firmware,
    press and hold the \ButtonLeft{} button while turning on the player.
    }

  \opt{clipv2,fuzev2,clipplus}
    {
        \note{Rockbox does not boot into the original firmware when powered by
        a USB connection. Older versions of Rockbox do not provide USB support.
        If you have such a version installed you need to manually boot into the
        original firmware for data transfer via USB.}
    }

  \opt{mrobe100}
    {
    Rockbox has a dual-boot feature. It loads the original firmware from
    the file \fname{/System/OF.mi4}. To boot into the original firmware,
    when the \dap{} is turned off, press the \ButtonPower{} button once and then
    a second time when the m:robe bootlogo (the headphone) appears. Hold the
    \ButtonPower{} button until you see the ``Loading original firmware...''
    message on the screen.
    }

  \opt{gigabeats}
    {
    Rockbox has a dual-boot feature. To boot into the original firmware,
    turn the \ButtonHold{} switch on just after turning on the \dap{}.
    To return to Rockbox, shutdown the \dap{}, then turn the battery switch
    on the bottom off then on again. Rockbox should now start.
    }

  \opt{cowond2}
    {
    Use \ButtonPower{} to boot the original \playerman{} firmware.
    }

  \opt{vibe500}
    {
    Rockbox has a dual-boot feature where it is possible to load the original firmware from
    the file \fname{/System/OF.mi4}. To boot into the original firmware press and release
    \ButtonPower{} and then immediately after the backlight turns on, press the \ButtonOK{}
    button and keep it pressed until the original firmware starts.
    }

  \opt{samsungyh}
    {
    Rockbox has a dual-boot feature. It loads the original firmware from
    the file \fname{/System/OF.mi4}. To boot into the original firmware, press and hold
    for awhile the \ButtonPlay{} button and then immediately after the Samsung logo appears,
    press the \ButtonLeft{} button and keep it pressed until the original firmware starts.
    }

  \opt{ondavx777}
    {
    Rockbox has a dual-boot feature where it is possible to load the original firmware from
    the file \fname{/SD/ccpmp.bin}. To boot into the original firmware press and release
    \ButtonPower{} immediately after the Rockbox Logo appear on the screen.
    }

  \opt{xduoox3}
    {
    Rockbox has a dual-boot feature. To boot into the original firmware,
    when the \dap{} is turned off, set the \ButtonLock{} switch to locked,
    and then press the \ButtonPower{} button.
    }

  \opt{fiiom3k,shanlingq1,erosqnative}
    {
    Rockbox has a dual-boot feature. To boot into the original firmware,
    hold \ActionBootOFPlayer{} when powering on the \dap{}.

    \nopt{erosqnative}{
      You can trigger a normal \playerman{} firmware update by holding
      \ActionBootOFRecovery{} when powering on the \dap{}.
      \warn{Updating the original firmware will \textbf{erase} the Rockbox
      bootloader.}
    }

    \subsection{Entering the recovery menu}
    You can access the Rockbox bootloader's ``recovery menu'' by holding
    \ActionBootRecoveryMenu{} while powering on the player. This menu can be
    used to connect your \dap{} over USB to transfer files, update the Rockbox
    bootloader, or revert to a bootloader you've previously backed up.
    }

  }
\subsection{Putting music on your \dap{}}

\opt{usb_hid}{
\note{Due to a bug in some OS X versions, the \dap{} can not be mounted, unless
    the USB HID feature is disabled. See \reference{ref:USB_HID} for more
    information.\newline
}
}

With the \dap{} connected to the computer as an MSC/UMS device (like a
USB Drive), music files can be put on the player via any standard file
transfer method that you would use to copy files between drives (e.g. Drag-and-Drop).
Files may be placed wherever you like on the \dap{}, but it is strongly
suggested \emph{NOT} to put them in the \fname{/.rockbox} folder and instead
put them in any other folder, e.g. \fname{/}, \fname{/music} or \fname{/audio}.
The default directory structure that is assumed by some parts of Rockbox
\opt{albumart}{%
    (album art searching, and missing-tag fallback in some WPSes) uses the
    parent directory of a song as the Album name, and the parent directory of
    that folder as the Artist name. WPSes may display information incorrectly if
    your files are not properly tagged, and you have your music organized in a
    way different than they assume when attempting to guess the Artist and Album
    names from your filetree. See \reference{ref:album_art} for the requirements
    for Album Art to work properly.
}%
\nopt{albumart}{%
    (missing-tag fallback in some WPSes) uses the parent directory of a song
    as the Album name, and the parent directory of that folder as the Artist
    name. WPSes may display
    information incorrectly if your files are not properly tagged, and you have
    your music organized in a way different than they assume when attempting to
    guess the Artist and Album names from your filetree.
}%
    See \reference{ref:Supportedaudioformats} for a list of supported audio
    formats.

\subsection{The first contact}

After you have first started the \dap{}, you'll be presented with the
\setting{Main Menu}. From this menu you can reach every function of Rockbox.
For more information, see \reference{ref:main_menu}. To browse the files
on your \dap{}, select \setting{Files} (see \reference{ref:file_browser}), and to
browse based on the meta-data\footnote{ID3 Tags, Vorbis
comments, etc.} of your audio files, select \setting{Database} (see
\reference{ref:database}).

\subsection{Basic controls}
Use \ActionStdNext{} and \ActionStdPrev{} to adjust the selection when browsing
through menus. \opt{wheel_acceleration}{
Note that scroll speed accelerates, the faster you rotate the wheel.}
Press \ActionStdOk{} to select an item. Selecting an audio file in the File Browser
will play it. The ``While playing screen'' (``WPS'', see \reference{ref:WPS}) will
be displayed, and the current playlist will be replaced with tracks from the current
directory. To go back to the bowser, either stop playback using the
\ActionWpsStop{} button, or press \ActionWpsBrowse{}, which keeps playback
running.
When browsing files, return to parent directories with \ActionTreeParentDirectory.

\subsection{Basic concepts}
\subsubsection{Playlists}
Rockbox is playlist-oriented. This means that every time you play an audio file,
a current playlist is generated.
You can modify the current playlist in many ways while playing, and also save
it to a file. But you are not forced to do so.
Playlists are covered in detail in \reference{ref:working_with_playlists}.

\subsubsection{Menu}
Menus are the primary way of interacting with Rockbox and offer quick access
to frequently-used functions.

\subsubsection{Context Menu}
In many places, a context menu can be accessed by pressing \ActionStdContext{}.
A context menu allows you to perform actions relevant to your current selection, and may
include more advanced commands.
In the file browser, the relevant selection may be a file (or directory).
On the WPS, it is the currently playing track. Some actions do not apply
to the selection, but relate to the screen itself, from which the context menu
was accessed. Playback settings accessible from the “While playing screen” are one such
example.

\section{Customising Rockbox}
Rockbox' User Interface can be customised using ``Themes''. Themes usually
only affect the visual appearance, but an advanced user can create a theme
that also changes various other settings like file view, LCD settings and
all other settings that can be modified using \fname{.cfg} files. This topic
is discussed in more detail in \reference{ref:manage_settings}.
The Rockbox distribution comes with some themes that should look nice on
your \dap{}.

\note{Some of the themes shipped with Rockbox need additional
fonts from the fonts package, so make sure you installed them.
Also, if you downloaded additional themes from the Internet make sure you
have the needed fonts installed as otherwise the theme may not display
properly.}

  \opt{usb_power}{
    \section{USB Charging}
    Your \dap{} will automatically charge when connected to USB. By default
    Rockbox will connect in mass storage mode to transfer files, but you can
    prevent this by holding down any button while plugging in the USB cable,
    or by changing the \setting{USB Mode} setting to \setting{Charge Only}.
    \nopt{fuzeplus}{
    \note{Be aware that holding a button may still perform its normal function,
    so it is recommended to use a button without harmful side effects, such as
    \ActionStdUsbCharge{}.}
    }
  }

% $Id$ %
\chapter{Browsing and playing}
\section{\label{ref:file_browser}File Browser}
\screenshot{rockbox_interface/images/ss-file-browser}{The file browser}{}
Rockbox lets you browse your music in either of two ways. The
\setting{File Browser} lets you navigate through the files and directories on
your \dap, entering directories and executing the default action on each file.
To help differentiate files, each file format is displayed with an icon.

The \setting{Database Browser}, on the other hand, allows you to navigate
through the music on your player using categories like album, artist, genre,
etc.

You can select whether to browse using the \setting{File Browser} or the
\setting{Database Browser} by selecting either \setting{Files} or
\setting{Database} in the \setting{Main Menu}.
If you choose the \setting{File Browser}, the \setting{Show Files} setting
lets you select what types of files you wish to view. See
\reference{ref:FileView} for more information on the \setting{Show Files}
setting.

\note{The \setting{File Browser} allows you to manipulate your files in ways
that are not available within the \setting{Database Browser}. Read more about
\setting{Database} in \reference{ref:database}. The remainder of this section
deals with the \setting{File Browser}.}

\opt{iriverh10,iriverh10_5gb}{\note{
If your \dap{} is a MTP model, the Music directory where all your music is stored
may be hidden in the \setting{File Browser}. This may be fixed by either
changing its properties (on a computer) to not hidden, or by changing
the \setting{Show Files} setting to all.
}}

\subsection{\label{ref:controls}File Browser Controls}
\begin{btnmap}
      \ActionStdPrev{}/\ActionStdNext{}
      \opt{HAVEREMOTEKEYMAP}{& \ActionRCStdPrev{}/\ActionRCStdNext{}}
         & Go to previous/next item in list. If you are on the first/last
           entry, the cursor will wrap to the last/first entry.\\
      %
      \opt{IRIVER_H100_PAD,IRIVER_H300_PAD}
        {
          \ButtonOn+\ButtonUp{}/ \ButtonDown
          \opt{HAVEREMOTEKEYMAP}{&
            \opt{IRIVER_RC_H100_PAD}{\ButtonRCSource{}/ \ButtonRCBitrate}
          }
          & Move one page up/down in the list.\\
        }
      \opt{IRIVER_H10_PAD}
        {
          \ButtonRew{}/ \ButtonFF
          & Move one page up/down in the list.\\
        }
      %
      \ActionTreeParentDirectory
      \opt{HAVEREMOTEKEYMAP}{& \ActionRCTreeParentDirectory}
      & Go to the parent directory.\\
      %
      \ActionTreeEnter
      \opt{HAVEREMOTEKEYMAP}{& \ActionRCTreeEnter}
      & Execute the default action on the selected file or enter a
        directory.\\
      %
      \ActionTreeWps
      \opt{HAVEREMOTEKEYMAP}{& \ActionRCTreeWps}
         & If there is an audio file playing, return to the
         \setting{While Playing Screen} (WPS) without stopping playback.\\
      %
      \nopt{player,SANSA_C200_PAD,erosqnative}%
        {%
          \ActionTreeStop
          \opt{HAVEREMOTEKEYMAP}{& \ActionRCTreeStop}
          & Stop audio playback.\\%
        }%
      %
      \ActionStdContext{}
      \opt{HAVEREMOTEKEYMAP}{& \ActionRCStdContext}
      & Enter the \setting{Context Menu}.\\
      %
      \ActionStdMenu{}
      \opt{HAVEREMOTEKEYMAP}{& \ActionRCStdMenu}
      & Enter the \setting{Main Menu}.\\
      %
      \nopt{erosqnative}{
        \opt{quickscreen}{
          \ActionStdQuickScreen
          \opt{HAVEREMOTEKEYMAP}{& \ActionRCStdQuickScreen}
          & Switch to the \setting{Quick Screen}
          (see \reference{ref:QuickScreen}). \\
        }
      }
      %
      \opt{SANSA_E200_PAD}{
        \ActionStdRec & Switch to the \setting{Recording Screen}.\\
      %
      }
      \nopt{touchscreen}{\opt{hotkey}{
        \ActionTreeHotkey
            &
        \opt{HAVEREMOTEKEYMAP}{
            &}
        Activate the \setting{Hotkey} function
        (see \reference{ref:Hotkeys}).
            \\
      }}
\end{btnmap}

\subsection{\label{ref:Contextmenu}\label{ref:PartIISectionFM}Context Menu}
\screenshot{rockbox_interface/images/ss-context-menu}{The Context Menu}{}

The \setting{Context Menu} allows you to perform certain operations on files or
directories.  To access the \setting{Context Menu}, position the selector over a file
or directory and access the context menu with \ActionStdContext{}.\\

\note{The \setting{Context Menu} is a context sensitive menu.  If the
\setting{Context Menu} is invoked on a file, it will display options available
for files.  If the \setting{Context Menu} is invoked on a directory,
it will display options for directories.\\}

The \setting{Context Menu} contains the following options (unless otherwise noted,
each option pertains both to files and directories):

\begin{description}
\item [View.]
  Displays the contents of the selected playlist file.
\item [Playing Next...]
  Enters the \setting{Playing Next Submenu} (see \reference{ref:playingnext_submenu}).
\item [Add to Playlist...]
  Enters the \setting{Add to Playlist Submenu} (see
  \reference{ref:addtoplaylist_submenu}).
\item [Rename.]
  This function lets the user modify the name of a file or directory.
\item [Cut.]
  Copies the name of the currently selected file or directory to the clipboard
  and marks it to be `cut'.
\item [Copy.]
  Copies the name of the currently selected file or directory to the clipboard
  and marks it to be `copied'.
\item [Paste.]
  Only visible if a file or directory name is on the clipboard. When selected
  it will move or copy the clipboard to the current directory.
\item [Delete.]
  Deletes the currently selected file. This option applies only to files, and
  not to directories. Rockbox will ask for confirmation before deleting a file.
  Press \ActionYesNoAccept{}
  to confirm deletion or any other key to cancel.
\item [Delete Directory.]
  Deletes the currently selected directory and all of the files and subdirectories
  it may contain. Deleted directories cannot be recovered. Use this feature with
  caution!
\item [Open with.]
  Runs a viewer plugin on the file. Normally, when a file is selected in Rockbox,
  Rockbox automatically detects the file type and runs the appropriate plugin.
  The \setting{Open With} function can be used to override the default action and
  select a viewer by hand.
  For example, this function can be used to view a text file
  even if the file has a non-standard extension (i.e., the file has an extension
  of something other than \fname{.txt}). See \reference{ref:Viewersplugins}
  for more details on viewers.
\item [Create Directory.]
  Create a new directory in the current directory on the disk.
\item [Properties / Show Track Info.]
  Shows properties such as size and the time and date of the last modification
  for the selected file. If used on a directory, the number of files and
  subdirectories will be shown, as well as the total size. If used on a supported
  audio file, its metadata will be displayed.
\opt{lcd_non-mono}{
\item [Set As Backdrop.]
  Set the selected \fname{bmp} file as background image. The bitmaps need to meet the
  conditions explained in \reference{ref:LoadingBackdrops}.
}
\item [Add to Shortcuts.]
  Adds a link to the selected item in the \fname{shortcuts.txt} file
  (see \reference{ref:MainMenuShortcuts}).
  If the file does not already exist it will be created in your Rockbox directory.
  Note that if you create a shortcut to a file, Rockbox will not open it upon
  selecting, but simply bring you to its location in the \setting{File Browser}.
\end{description}

\subsubsection{\label{ref:SetAs}Set As...}
\begin{description}
  \item [Playlist Directory.]
     Set as default playlist directory. Playlists from this directory will appear
     when you select \setting{Playlists} from the main menu
     (see \reference{ref:playlistoptions}).
  \opt{tagcache}{
  \item [Database Directory.]
     Rockbox usually stores database files in the \fname{/.rockbox} folder.
     You can choose another location for the database using this setting.
     This is mainly useful for multiboot targets, so the same database can
     be shared among several builds without needing to rebuild it each time.
  }
  \opt{recording}{
    \item [Recording Directory.]
     Save recordings in the selected directory.
  }
  \item [Start Directory.]
   Set as default start directory for the File Browser.
   \note{If you have \setting{Auto-Change Directory} and
   \setting{Constrain Auto-Change} enabled, the directories returned will
   be constrained to the directory you have chosen here and those below it.
   See \reference{ref:ConstrainAutoChange}}
\end{description}

\subsection{\label{sec:virtual_keyboard}Virtual Keyboard}
\screenshot{rockbox_interface/images/ss-virtual-keyboard}{The virtual keyboard}{}
This is the virtual keyboard that is used when entering text in Rockbox, for
example when renaming a file or creating a new directory.
The virtual keyboard can be easily changed by making a text file
with the required layout. More information on how to achieve this can be found
on the Rockbox website at \wikilink{LoadableKeyboardLayouts}.

\opt{morse_input}{
  Also you can switch to Morse code input mode by changing the
  \setting{Use Morse Code Input} setting%
  \opt{IRIVER_H100_PAD,IRIVER_H300_PAD,IPOD_4G_PAD,IPOD_3G_PAD,IRIVER_H10_PAD%
      ,GIGABEAT_PAD,GIGABEAT_S_PAD,MROBE100_PAD,SANSA_E200_PAD,PBELL_VIBE500_PAD%
      ,SANSA_FUZEPLUS_PAD,SAMSUNG_YH92X_PAD,SAMSUNG_YH820_PAD}
    { or by pressing \ActionKbdMorseInput{} in the virtual keyboard}%
  .}

% no "Actions" yet in the Player's virtual keyboard

\note{When the cursor is on the input line, \ActionKbdSelect{} deletes the preceding character}

\begin{btnmap}
    \opt{IRIVER_H100_PAD,IRIVER_H300_PAD,GIGABEAT_PAD,GIGABEAT_S_PAD%
        ,MROBE100_PAD,SANSA_E200_PAD,SANSA_FUZE_PAD,SANSA_C200_PAD,SANSA_FUZEPLUS_PAD%
        ,SAMSUNG_YH820_PAD}{
        \ActionKbdCursorLeft{} / \ActionKbdCursorRight
            &
        \opt{HAVEREMOTEKEYMAP}{\ActionRCKbdCursorLeft{} / \ActionRCKbdCursorRight
            &}
        Move the line cursor within the text line.
            \\
        %
        \ActionKbdBackSpace
            &
        \opt{HAVEREMOTEKEYMAP}{
            &}
        Delete the character before the line cursor.
            \\
    }%
    \ActionKbdLeft{} / \ActionKbdRight
        &
    \opt{HAVEREMOTEKEYMAP}{\ActionRCKbdLeft{} / \ActionRCKbdRight
        &}
    Move the cursor on the virtual keyboard.
    If you move out of the picker area, you get the previous/next page of
    characters (if there is more than one).
        \\
    %
    \ActionKbdUp{} / \ActionKbdDown
        &
    \opt{HAVEREMOTEKEYMAP}{\ActionRCKbdUp{} / \ActionRCKbdDown
        &}
    Move the cursor on the virtual keyboard.
    If you move out of the picker area you get to the line edit mode.
        \\
    %
    \nopt{IPOD_3G_PAD,IPOD_4G_PAD,IRIVER_H10_PAD,PBELL_VIBE500_PAD%
         ,SANSA_FUZEPLUS_PAD,SAMSUNG_YH92X_PAD,SAMSUNG_YH820_PAD}{
        \ActionKbdPageFlip
            &
        \opt{HAVEREMOTEKEYMAP}{\ActionRCKbdPageFlip
            &}
        Flip to the next page of characters (if there is more than one).
            \\
    }
    %
    \ActionKbdSelect
        &
    \opt{HAVEREMOTEKEYMAP}{\ActionRCKbdSelect
        &}
    Insert the selected keyboard letter at the current line cursor position.
        \\
    %
    \ActionKbdDone
        &
    \opt{HAVEREMOTEKEYMAP}{\ActionRCKbdDone
        &}
    Exit the virtual keyboard and save any changes.
        \\
    %
    \ActionKbdAbort
        &
    \opt{HAVEREMOTEKEYMAP}{\ActionRCKbdAbort
        &}
    Exit the virtual keyboard without saving any changes.
        \\
% to be done - create a separate section for morse imput and update the info
      \opt{morse_input}{
        \opt{IRIVER_H100_PAD,IRIVER_H300_PAD,GIGABEAT_PAD,GIGABEAT_S_PAD,MROBE100_PADD%
            ,SANSA_E200_PA,IPOD_4G_PAD,IPOD_3G_PAD,IRIVER_H10_PAD,PBELL_VIBE500_PAD%
            ,SAMSUNG_YH92X_PAD,SAMSUNG_YH820_PAD}{
          \ActionKbdMorseInput
          \opt{HAVEREMOTEKEYMAP}{& \ActionRCKbdMorseInput}
          & Toggle keyboard input mode and Morse code input mode. \\}
        %
        \ActionKbdMorseSelect
        \opt{HAVEREMOTEKEYMAP}{& \ActionRCKbdMorseSelect}
        & Tap to select a character in Morse code input mode. \\
      }
\end{btnmap}

% $Id$ %
\section{\label{ref:database}Database}

\subsection{Introduction}
This chapter describes the Rockbox music database system. Using the information
contained in the tags (ID3v1, ID3v2, Vorbis Comments, Apev2, etc.) in your
audio files, Rockbox builds and maintains a database of the music
files on your player and allows you to browse them by Artist, Album, Genre, 
Song Name, etc.  The criteria the database uses to sort the songs can be completely
 customised. More information on how to achieve this can be found on the Rockbox
 website at \wikilink{DataBase}. 

\subsection{Initializing the Database}
The first time you use the database, Rockbox will scan your disk for audio files.
This can take quite a while depending on the number of files on your \dap{}.
This scan happens in the background, so you can choose to return to the
Main Menu and continue to listen to music.
If you shut down your player, the scan will continue next time you turn it on.
After the scan is finished you may be prompted to restart your \dap{} before
you can use the database.

\subsubsection{Ignoring Directories During Database Initialization}

You may have directories on your \dap{} whose contents should not be added
to the database. Placing a file named \fname{database.ignore} in a directory
will exclude the files in that directory and all its subdirectories from
scanning their tags and adding them to the database. This will speed up the
database initialization.

If a subdirectory of an `ignored' directory should still be scanned, place a
file named \fname{database.unignore} in it. The files in that directory and
its subdirectories will be scanned and added to the database.

\subsubsection{Issues During Database Commit}

You may have files on your \dap{} whose contents might not be displayed
correctly or even crash the database.
Placing a file named \fname{/.rockbox/database\_commit.ignore}
will prevent the device from committing the database automatically
you can manually commit the database using the db\_commit plugin in APPS
with logging

\subsection{\label{ref:databasemenu}The Database Menu}

\begin{description}
  \opt{tc_ramcache}{
  \item[Load To RAM]
    The database can either be kept on \disk{} (to save memory), or
    loaded into RAM (for fast browsing). Setting this to \setting{Yes} loads
    the database to RAM, beginning with the next reboot, allowing faster browsing and
    searching. Setting this option to \setting{No} keeps the database on the \disk{},
    meaning slower browsing but it does not use extra RAM and saves some battery on
    boot up.

    \opt{HAVE_DISK_STORAGE}{
      If you browse your music frequently using the database, you should
      load to RAM, as this will reduce the overall battery consumption because
      the disk will not need to spin on each search.
    }

    \note{When Load to RAM is turned on, and the directory cache (see
    \reference{ref:dircache}) is enabled as well, it may take an unexpectedly long amount
    of time for disk activity to wind down after booting, depending on your library size
    and player.

    This can be mitigated by choosing the \setting{Quick} option instead, which causes
    the database to ignore cached file references. In that case, you should expect brief
    moments of disk activity whenever the path for a database entry has to be retrieved.

    Setting this to \setting{On} may be preferable for reducing disk accesses if you plan to
    take advantage of \setting{Auto Update}, have enabled \setting{Gather Runtime Data}
    (see below for both), enabled \setting{Automatic resume} (see
    \reference{ref:Autoresumeconfigactual}), or use a WPS that displays multiple upcoming
    tracks from the current playlist. In the latter case, metadata will not be displayed
    for those tracks otherwise.}
  }

\item[Auto Update]
  If \setting{Auto update} is set to \setting{on}, each time the \dap{}
  boots, the database will automatically be updated.

\item[Initialize Now]
  You can force Rockbox to rescan your disk for tagged files by
  using the \setting{Initialize Now} function in the \setting{Database
    Menu}.
  \warn{\setting{Initialize Now} removes all database files (removing
    runtimedb data also) and rebuilds the database from scratch.}

\item[Update Now]
  \setting{Update now} causes the database to detect new and deleted files
    \note{Unlike the \setting{Auto Update} function, \setting{Update Now}
      will update the database regardless of whether the \setting{Directory Cache}
      is enabled. Thus, an update using \setting{Update now} may take a long
      time.
  }
  Unlike \setting{Initialize Now}, the \setting{Update Now} function
  does not remove runtime database information.
  
\item[Gather Runtime Data]
  When enabled, rockbox will record how often and how long a track is being played, 
  when it was last played and its rating. This information can be displayed in
  the WPS and is used in the database browser to, for example, show the most played, 
  unplayed and most recently played tracks.
  
\item[Export Modifications]
  This allows for the runtime data to be exported to the file \\
  \fname{/.rockbox/database\_changelog.txt}, which backs up the runtime data in
  ASCII format. This is needed when database structures change, because new
  code cannot read old database code. But, all modifications
  exported to ASCII format should be readable by all database versions.
  
\item[Import Modifications.]
  Allows the \fname{/.rockbox/database\_changelog.txt} backup to be 
  conveniently loaded into the database. If \setting{Auto Update} is
  enabled this is performed automatically when the database is initialized.

\item[Select directories to scan.]
  The database normally scans all directories for audio files. This setting
  allows you to limit the scan to a specified list of directories, so only
  files contained in one of these directories will be added to the database.
  Scanning is recursive -- all subdirectories of a selected directory will
  be scanned as well.

\end{description}

\subsection{Using the Database}
Once the database has been initialized, you can browse your music 
by Artist, Album, Genre, Song Name, etc.  To use the database, go to the
 \setting{Main Menu} and select \setting{Database}.\\

\note{You may need to increase the value of the \setting{Max Entries in File
Browser} setting (\setting{Settings $\rightarrow$ General Settings
$\rightarrow$ System $\rightarrow$ Limits}) in order to view long lists of
tracks in the ID3 database browser.\\

To "turn off" the database, do not initialize it, and do not load it to RAM.}%

If the number of tracks you are attempting to play exceeds the value of the
\setting{Max Playlist Size} setting (\setting{Settings $\rightarrow$ General
Settings $\rightarrow$ System $\rightarrow$ Limits}), a randomly chosen
subset of tracks is added to the current playlist instead. This allows you
to play random tracks from your whole library without having to increase any
limits that may degrade performance.

\note{For your convenience, a shortcut button "Shuffle Songs" is available directly
from the \setting{Database} menu to create and start a mix with all of your
existing music tracks.}

\begin{table}
  \begin{rbtabular}{.75\textwidth}{XXX}%
  {\textbf{Tag}   & \textbf{Type}  & \textbf{Origin}}{}{}
  filename              & string    & system \\ 
  album                 & string    & id tag \\
  albumartist           & string    & id tag \\
  artist                & string    & id tag \\
  comment               & string    & id tag \\
  composer              & string    & id tag \\
  genre                 & string    & id tag \\
  grouping              & string    & id tag \\
  title                 & string    & id tag \\
  bitrate               & numeric   & id tag \\
  discnum               & numeric   & id tag \\
  year                  & numeric   & id tag \\
  tracknum              & numeric   & id tag/filename \\
  autoscore             & numeric   & runtime db \\
  lastplayed            & numeric   & runtime db \\
  playcount             & numeric   & runtime db \\
  Pm (play time -- min)  & numeric   & runtime db \\
  Ps (play time -- sec)  & numeric   & runtime db \\
  rating                & numeric   & runtime db \\
  commitid              & numeric   & system \\
  entryage              & numeric   & system \\
  length                & numeric   & system \\
  Lm (track len -- min)  & numeric   & system \\
  Ls (track len -- sec)  & numeric   & system \\
  \end{rbtabular}
\end{table}

% $Id$ %
\section{\label{ref:WPS}While Playing Screen}
The While Playing Screen (WPS) displays various pieces of information about the
currently playing audio file.
%
The appearance of the WPS can be configured using WPS configuration files.
The items shown depend on your configuration -- all items can be turned on
or off independently. Refer to \reference{ref:wps_tags} for details on how
to change the display of the WPS.
\begin{itemize}
\item Status bar: The Status bar shows Battery level, charger status,
  volume, play mode, repeat mode, shuffle mode\opt{rtc}{ and clock}.
  In contrast to all other items, the status bar is always at the top of
  the screen.
\item (Scrolling) path and filename of the current song.
\item The ID3 track name.
\item The ID3 album name.
\item The ID3 artist name.
\item Bit rate. VBR files display average bitrate and ``(avg)''
\item Elapsed and total time.
\item A slidebar progress meter representing where in the song you are.
\item Peak meter.
\end{itemize}
%

See \reference{ref:ConfiguringtheWPS} for details of customising
your WPS (While Playing Screen).


\subsection{\label{ref:WPS_Key_Controls}WPS Key Controls}

  \begin{btnmap}
      \ActionWpsVolUp{} / \ActionWpsVolDown
      \opt{HAVEREMOTEKEYMAP}{& \ActionRCWpsVolUp{} / \ActionRCWpsVolDown}
      & Volume up/down.\\
      %
      \ActionWpsSkipPrev
       \opt{HAVEREMOTEKEYMAP}{& \ActionRCWpsSkipPrev}
      & Go to beginning of track, or if pressed while in the
        first seconds of a track, go to the previous track.\\
      %
      \ActionWpsSeekBack
      \opt{HAVEREMOTEKEYMAP}{& \ActionRCWpsSeekBack}
      & Rewind in track.\\
      %
      \ActionWpsSkipNext
      \opt{HAVEREMOTEKEYMAP}{& \ActionRCWpsSkipNext}
      & Go to the next track.\\
      %
      \ActionWpsSeekFwd
      \opt{HAVEREMOTEKEYMAP}{& \ActionRCWpsSeekFwd}
      & Fast forward in track.\\
      %
      \ActionWpsPlay
      \opt{HAVEREMOTEKEYMAP}{& \ActionRCWpsPlay}
      & Toggle play/pause.\\
      %
      \ActionWpsStop
      \opt{HAVEREMOTEKEYMAP}{& \ActionRCWpsStop}
      & Stop playback.\\
      %
      \ActionWpsBrowse
      \opt{HAVEREMOTEKEYMAP}{& \ActionRCWpsBrowse}
      & Return to the \setting{File Browser} / \setting{Database} / \setting{Playlists}.\\
      %
      \ActionWpsContext
      \opt{HAVEREMOTEKEYMAP}{& \ActionRCWpsContext}
      & Enter \setting{WPS Context Menu}.\\
      %
      \ActionWpsMenu
      \opt{HAVEREMOTEKEYMAP}{& \ActionRCWpsMenu}
      & Enter \setting{Main Menu}%
      .\\%
      %
      \opt{quickscreen}{%
        \ActionWpsQuickScreen
        \opt{HAVEREMOTEKEYMAP}{& \ActionRCWpsQuickScreen}
          & Switch to the \setting{Quick Screen}
          (see \reference{ref:QuickScreen}). \\}%
      %
      % software hold targets
      \nopt{hold_button}{%
          \opt{SANSA_CLIP_PAD}{\ButtonHome+\ButtonSelect}
          \opt{SANSA_FUZEPLUS_PAD,AIGO_EROSQ_PAD}{\ButtonPower}
          \opt{RG_NANO_PAD}{\ButtonStart+\ButtonFN}
          & Key lock (software hold switch) on/off.\\
      }%
      \opt{RG_NANO_PAD}{\ButtonStart+\ButtonFN}
      & Key lock (software hold switch) on/off.\\
      % We explicitly list all the appropriate targets here and do no condition
      % on the 'pitchscreen' feature since some players have the feature but do
      % not have the button to go from the WPS to the pitch screen.
      \opt{IRIVER_H100_PAD,IRIVER_H300_PAD,IRIVER_H10_PAD,MROBE100_PAD%
          ,GIGABEAT_PAD,GIGABEAT_S_PAD,SANSA_E200_PAD,SANSA_C200_PAD,SANSA_FUZEPLUS_PAD}{%
        \ActionWpsPitchScreen
        \opt{HAVEREMOTEKEYMAP}{& \ActionRCWpsPitchScreen}
          & Show \setting{Pitch Screen} (see \reference{sec:pitchscreen}).\\%
      }%
      \opt{GIGABEAT_PAD,GIGABEAT_S_PAD,SANSA_CLIP_PAD,MROBE100_PAD,PBELL_VIBE500_PAD%
          ,SAMSUNG_YH92X_PAD,SAMSUNG_YH820_PAD,XDUOO_X3_PAD}{%
        \ActionWpsPlaylist
        \opt{HAVEREMOTEKEYMAP}{&}
          & Show current \setting{Playlist}.\\%
      }%
      \opt{IRIVER_H100_PAD,IRIVER_H300_PAD,IRIVER_H10_PAD%
          ,SANSA_E200_PAD,SANSA_C200_PAD,SANSA_FUZEPLUS_PAD}{%
        \ActionWpsIdThreeScreen
          \opt{HAVEREMOTEKEYMAP}{& \ActionRCWpsIdThreeScreen}
          & Enter \setting{ID3 Viewer}.\\%
      }%
      \opt{hotkey}{%
        \ActionWpsHotkey \opt{HAVEREMOTEKEYMAP}{& }
        & Activate the \setting{Hotkey} function (see \reference{ref:Hotkeys}).\\
      }
      \opt{ab_repeat_buttons}{%
         \ActionWpsAbSetBNextDir{} or }%
         % not all targets have the above action defined but the one below works on all
      \nopt{erosqnative}{
        Short \ActionWpsSkipNext{} + Long \ActionWpsSkipNext
        \opt{HAVEREMOTEKEYMAP}{
          &
          \opt{IRIVER_RC_H100_PAD}{\ActionRCWpsAbSetBNextDir{} or}
          Short \ActionRCWpsSkipNext{} + Long \ActionRCWpsSkipNext}
        & Skip to the next directory.\\
        %
        \opt{ab_repeat_buttons}{%
           \ActionWpsAbSetAPrevDir{} or }%
        Short \ActionWpsSkipPrev{} + Long \ActionWpsSkipPrev
        \opt{HAVEREMOTEKEYMAP}{
          &
          \opt{IRIVER_RC_H100_PAD}{\ActionRCWpsAbSetAPrevDir{} or}
          Short \ActionRCWpsSkipPrev{} + Long \ActionRCWpsSkipPrev}
        & Skip to the previous directory.\\
      }
      %
      \opt{SANSA_E200_PAD,SANSA_C200_PAD,IRIVER_H100_PAD,IRIVER_H300_PAD}{
        \ActionStdRec
          \opt{HAVEREMOTEKEYMAP}{&}
          & Switch to the \setting{Recording Screen}.\\
      }%
  \end{btnmap}


\subsection{\label{ref:peak_meter}Peak Meter}
The peak meter can be displayed on the While Playing Screen and consists of
several indicators.
\opt{recording}{
  For a picture of the peak meter, please see the While
  Recording Screen in \reference{ref:while_recording_screen}.
}
\opt{ipodvideo}{
  \note{Especially the \playerman{} \playertype{}'s performance and battery runtime
   suffers when this feature is enabled. For this \dap{} it is highly recommended
   to not use peak meter.}
}

\begin{description}
\item [The bar:]
  This is the wide horizontal bar. It represents the current volume value.
\item [The peak indicator:]
  This is a little vertical line at the right end of the bar. It indicates
  the peak volume value that occurred recently.
\item [The clip indicator:]
  This is a little black block that is displayed at the very right of the
  scale when an overflow occurs. It usually does not show up during normal
  playback unless you play an audio file that is distorted heavily.
  \opt{recording}{
    If you encounter clipping while recording, your recording will sound distorted.
    You should lower the gain.
  }
  \note{Note that the clip detection is not very precise.
   Clipping might occur without being indicated.}
\item [The scale:]
  Between the indicators of the right and left channel there are little dots.
  These dots represent important volume values. In linear mode each dot is a
  10\% mark. In dBFS mode the dots represent the following values (from right
  to left): 0~dB, {}-3~dB, {}-6~dB, {}-9~dB, {}-12~dB, {}-18~dB, {}-24~dB, {}-30~dB,
  {}-40~dB, {}-50~dB, {}-60~dB.
\end{description}

\subsection{\label{sec:contextmenu}The WPS Context Menu}
Like the context menu for the \setting{File Browser}, the \setting{WPS Context Menu}
allows you quick access to some often-used functions.

\subsubsection{Current Playlist}
The \setting{Current Playlist} submenu allows you to view, save, search, reshuffle,
and display the play time of the current playlist. These and other operations
are detailed in \reference{ref:working_with_playlists}. To change settings for
the \setting{Playlist Viewer} press \ActionStdContext{} while viewing the
current playlist to bring up the \setting{Playlist Viewer Menu}. In this
menu, you can find the \setting{Playlist Viewer Settings}.

\paragraph{Playlist Viewer Settings}
  \begin{description}
    \item[Show Icons.] This toggles display of the icon for the currently
    selected playlist entry and the icon for moving a playlist entry
    \item[Show Indices.] This toggles display of the line numbering for
       the playlist
    \item[Track Display.] This toggles between filename only and full path
       for playlist entries
  \end{description}


\subsubsection{Add to Playlist...}
  \begin{description}
    \item [Add to Existing Playlist.] Adds the currently playing file to a playlist.
    Select the playlist you want the file to be added to and it will get
    appended to that playlist.
    \item [Add to New Playlist.] Similar to the previous entry this will
    add the currently playing track to a playlist. You need to enter a name
    for the new playlist first.
  \end{description}

\subsubsection{Sound Settings}
This is a shortcut to the \setting{Sound Settings Menu}, where you can configure volume,
bass, treble, and other settings affecting the sound of your music.
See \reference{ref:configure_rockbox_sound} for more information.

\subsubsection{Playback Settings}
This is a shortcut to the \setting{Playback Settings Menu}, where you can configure shuffle,
repeat, party mode, skip length and other settings affecting the playback of your music.

\subsubsection{Rating}
The menu entry is only shown if \setting{Gather Runtime Information} is
enabled. It allows the assignment of a personal rating value (0 -- 10)
to a track which can be displayed in the WPS and used in the Database
browser. The value wraps at 10.

\subsubsection{\label{ref:createbookmark}Bookmarks}
Create a bookmark in the currently playing track, or display existing entries.
Bookmarks can only be used with directories, or playlists stored on
disk. You'll be asked to provide the name for a file to save the playlist to,
if necessary.

\note{If you've enabled options for queuing tracks (see \reference{ref:queuing}), keep
in mind that those are not saved to a playlist file.
When saving, Rockbox will offer to remove queued tracks, so that bookmarking the
current playlist becomes possible.}

Refer to \reference{ref:Bookmarkconfigactual} for more information on bookmarks.

\subsubsection{\label{ref:trackinfoviewer}Show Track Info}
\screenshot{rockbox_interface/images/ss-id3-viewer}{The track info viewer}{}
Provides a detailed view of all the identity information about the current
or selected track. This info is known as meta data and is stored in audio
file formats to keep information on artist, album etc. To get to this screen
from the WPS, %
\opt{IRIVER_H100_PAD,IRIVER_H300_PAD,IRIVER_H10_PAD,%
      SANSA_C200_PAD,SANSA_E200_PAD,SANSA_FUZE_PAD,SANSA_FUZEPLUS_PAD}{
  press \ActionWpsIdThreeScreen. }%
\opt{IPOD_4G_PAD,IPOD_3G_PAD,IAUDIO_X5_PAD,IAUDIO_M3_PAD,FIIO_M3K_PAD,%
      GIGABEAT_PAD,GIGABEAT_S_PAD,MROBE100_PAD,SANSA_CLIP_PAD,PBELL_VIBE500_PAD,%
      MPIO_HD200_PAD,MPIO_HD300_PAD,SAMSUNG_YH92X_PAD,SAMSUNG_YH820_PAD,XDUOO_X3_PAD}%
      {press \ActionWpsContext{} to access the
      \setting{Context Menu}. Then select \setting{Show Track Info}. The same steps
      work in the Playlist Viewer as well.}

\subsubsection{Open With...}
This \setting{Open With} function is the same as the \setting{Open With}
function in the file browser's \setting{Context Menu}.

\subsubsection{Delete}
Delete the currently playing file. The file will be deleted but the playback
of the file will not stop immediately. Instead, the part of the file that
has already been buffered (i.e. read into the \daps\ memory) will be played.
This may even be the whole track.

\opt{pitchscreen}{
  \subsubsection{\label{sec:pitchscreen}Pitch}

  The \setting{Pitch Screen} allows you to change the rate of playback
  (i.e. the playback speed and at the same time the pitch) of your
  \dap.  The rate value can be adjusted
  between 50\% and 200\%. 50\% means half the normal playback speed and a
  pitch that is an octave lower than the normal pitch. 200\% means double
  playback speed and a pitch that is an octave higher than the normal pitch.

  The rate can be changed in two modes: procentual and semitone.
  Initially, procentual mode is active.

    If you've enabled the \setting{Timestretch} option in
    \setting{Sound Settings} and have since rebooted, you can also use
    timestretch mode. This allows you to change the playback speed
    without affecting the pitch, and vice versa.

    In timestretch mode there are separate displays for pitch and
    speed, and each can be altered independently.  Due to the
    limitations of the algorithm, speed is limited to be between 35\%
    and 250\% of the current pitch value.  Pitch must maintain the
    same ratio as well as remain between 50\% and 200\%.

  The value of the rate, pitch and speed
  is persistent, i.e. when the \dap\ is turned on it will
  always be set to the last value set by \setting{Pitch Screen} or Bookmarks.
  Selecting \setting{Pitch} again will now display a menu with
  \setting{Pitch} and \setting{Reset Setting}.
  Selecting \setting{Reset Setting} will reset the pitch to 100\% now and at next boot.
  However the rate, pitch and speed information is stored in any bookmarks
  you may create (see \reference{ref:Bookmarkconfigactual})
  and will be restored upon playing back those bookmarks.

  \begin{btnmap}
    \ActionPsToggleMode
    \opt{HAVEREMOTEKEYMAP}{& \ActionRCPsToggleMode}
    & Toggle pitch changing mode (cycle through all available modes).\\
    %
    \ActionPsIncSmall{} / \ActionPsDecSmall
    \opt{HAVEREMOTEKEYMAP}{& \ActionRCPsIncSmall{} / \ActionRCPsDecSmall}
    & Increase~/ Decrease pitch by 0.1\% (in procentual mode) or 0.1
      semitone (in semitone mode).\\
    %
    \nopt{PBELL_VIBE500_PAD}{ % there is no long scroll up or down because of slide
    \ActionPsIncBig{} / \ActionPsDecBig
    \opt{HAVEREMOTEKEYMAP}{& \ActionRCPsIncBig{} / \ActionRCPsDecBig}
    & Increase~/ Decrease pitch by 1\% (in procentual mode) or a semitone
      (in semitone mode).\\
    }
    %
    \ActionPsNudgeLeft{} / \ActionPsNudgeRight
    \opt{HAVEREMOTEKEYMAP}{& \ActionRCPsNudgeLeft{} / \ActionRCPsNudgeRight}
    & Temporarily change pitch by 2\% (beatmatch), or modify speed (in timestretch mode).\\
    %
    \ActionPsReset
    \opt{HAVEREMOTEKEYMAP}{& \ActionRCPsReset}
    & Reset pitch and speed to 100\%. \\
    %
    \ActionPsExit
    \opt{HAVEREMOTEKEYMAP}{& \ActionRCPsExit}
    & Leave the \setting{Pitch Screen}. \\
    %
  \end{btnmap}

}

\subsubsection{View Album Art}
Display artwork for the currently playing track.


%Include playlist section
% $Id$ %
\chapter{The Main Menu}
\section{\label{ref:main_menu}Introducing the Main Menu}
\screenshot{main_menu/images/ss-main-menu}{The main menu}{}
The \setting{Main Menu} is the screen from which all of the Rockbox functions
can be accessed. This is the first screen you will see when starting Rockbox.
To return to the \setting{Main Menu}, press the \ActionStdMenu{} button.

All settings are stored on the unit. However, Rockbox does not access 
the \disk{} solely for the purpose of saving settings. Instead, Rockbox will
save settings when it accesses the \disk{} the next time, for example when 
refilling the music buffer or navigating through the \setting{File Browser}.
Changes to settings may therefore not be saved unless the \dap{} is shut down
safely (see \reference{ref:Safeshutdown}).

\section{Navigating the Main Menu}
  \begin{btnmap}
    \ActionStdNext
        &
    \opt{HAVEREMOTEKEYMAP}{\ActionRCStdNext
        &}
    Select the next option in the menu.\newline
    Inside a setting, increase the value or choose next option.
        \\
    %
    \ActionStdPrev
        &
    \opt{HAVEREMOTEKEYMAP}{\ActionRCStdPrev
        &}
    Select the previous option in the menu.\newline
    Inside a setting,decrease the value or choose previous option.
        \\
    %
    \ActionStdOk
        &
    \opt{HAVEREMOTEKEYMAP}{\ActionRCStdOk
        &}
    Select option.
        \\
    %
    \ActionStdCancel
        &
    \opt{HAVEREMOTEKEYMAP}{\ActionRCStdCancel
        &}
    Exit menu or setting, or move to parent menu.
        \\
  \end{btnmap}

\section {Recent Bookmarks}
\screenshot{main_menu/images/ss-list-bookmarks}%
{The list bookmarks screen}{}
If the \setting{Save a list of recently created bookmarks} option is enabled 
then you can view a list of several recent bookmarks here and select one to 
jump straight to that track.\\*

 \note{A track launched from the file browser can be bookmarked as-is. For
       tracks launched via the database, or for modified playlists, you're
       asked to save the playlist to a file when creating a bookmark.\\*} 

  \begin{btnmap}
    \ActionStdNext
    \opt{HAVEREMOTEKEYMAP}{& \ActionRCStdNext}
    & Select the next bookmark.\\
    %
    \ActionStdPrev
    \opt{HAVEREMOTEKEYMAP}{& \ActionRCStdPrev}
    & Select the previous bookmark.\\
    %
    \ActionStdOk
    \opt{HAVEREMOTEKEYMAP}{& \ActionRCStdOk}
    & Resume from the selected bookmark.\\
    %
    \ActionStdCancel
    \opt{HAVEREMOTEKEYMAP}{& \ActionRCStdCancel}
    & Exit Recent Bookmark menu.\\
    %
    \nopt{GIGABEAT_S_PAD}{\ActionBmDelete
    \opt{HAVEREMOTEKEYMAP}{& \ActionRCBmDelete}
    & Delete the currently selected bookmark.\\}
    %
    \ActionStdContext
    \opt{HAVEREMOTEKEYMAP}{& \ActionRCStdContext}  
    & Enter the context menu for the selected bookmark.\\
  \end{btnmap}

There are two options in the context menu:\\*
  
  \setting{Resume} will commence playback of the currently selected bookmark entry.
  
  \setting{Delete} will remove the currently selected bookmark entry from the list.\\*
  
This entry is not shown in the \setting{Main Menu} when the option is off
(the default setting).  See \reference{ref:Bookmarkconfigactual} 
for more details on configuring bookmarking in Rockbox.

\section{Files}
Browse the files on your \dap{} (see \reference{ref:file_browser}).

\section{Database}
Browse by the meta-data in your audio files (see \reference{ref:database}).

\section{Now Playing/Resume Playback}
Go to the \setting{While Playing Screen} and resume if music playback is
stopped or paused and there is something to resume (see \reference{ref:WPS}).

\section{Settings}

The \setting{Settings} menu allows you to set or adjust many parameters that
affect the way your \dap{} works. There are many submenus for different
parameter areas. Every time you are setting a value of a parameter, and that
value is selected from a list of some predefined available values, you can press
\ActionStdContext, and the selection cursor will jump to the default value for
the parameter. You can then confirm or cancel the value. This is useful if you
have changed the value of the parameter from the default to some other value and
would like to restore the default value.

\subsection{Sound Settings}
The \setting{Sound Settings} menu offers a selection of sound properties you may 
change to customise your listening experience. The details of this menu are covered
in \reference{ref:configure_rockbox_sound}.

\subsection{Playback Settings}
The \setting{Playback Settings} menu allows you to configure settings related
to audio playback. The details of this menu are covered
in \reference{ref:configure_rockbox_playback}.

\subsection{General Settings}
The \setting{General Settings} menu allows you to customise the way Rockbox looks 
and the way it plays music. The details of this menu are covered in
\reference{ref:configure_rockbox_general}.

\subsection{Theme Settings}
The \setting{Theme Settings} menu contains options that control the visual
appearance of Rockbox. The details of this menu are covered in
\reference{ref:configure_rockbox_themes}.

\opt{recording}{
\subsection{Recording Settings}
The \setting{Recording Settings} menu allows you to configure settings related
to recording. The details of this menu are covered in detail in
\reference{ref:Recordingsettings}.
}

\subsection{Manage Settings}
The \setting{Manage Settings} option allows the saving and re-loading of user 
configuration settings, browsing the hard drive for alternate firmwares, and finally
resetting your \dap{} back to initial configuration.
%
The details of this menu are covered in
\reference{ref:manage_settings}.

\opt{recording}{\input{main_menu/recording_screen.tex}}

\opt{radio}{\input{main_menu/fmradio.tex}}

\section{\label{ref:playlistoptions}Playlist Catalogue}
  This menu provides a simple interface to maintain
  several playlists (see \reference{ref:working_with_playlists}). Playlists can be created in
  three ways. Playing a file in a directory causes all the files in it
  to be placed in a playlist. Playlists can be created manually by
  either using the  \setting{Context Menu} (see \reference{ref:Contextmenu}) or using
  the \setting{Current Playlist} menu. Both automatically and manually created
  playlists can be edited using this menu.

\subsection{\label{ref:playlistcatalogue_contextmenu}The Playlist Catalogue Context Menu}

\begin{description}
\item[Create Playlist:]
  Rockbox will create a playlist with all tracks from the directory that is
currently selected in the \setting{File Browser}, including
all of its sub-directories. If you have selected the root level or your Playlist Catalogue directory,
Rockbox will instead create a playlist with all tracks from all directories. The playlist
will be saved in the Playlist Catalogue directory.

\item[View Current Playlist:]
  Displays the contents of the current dynamic playlist.

\item[Save Current Playlist:]
  Saves the current dynamic playlist, excluding queued tracks, to the
specified file. Asks for confirmation whether to remove queued tracks
from the current playlist, so that bookmarks can be created
(see \reference{ref:createbookmark}).
If no path is provided then playlist is saved to the current directory.

\item[Reset Playlist Catalogue Directory:]
  Will reset the default location for playlists to the \fname{/Playlists}
directory. Playlists stored in other locations are not included in the catalogue.

\end{description}

\section{Plugins}
  With this option you can load and run various plugins that have been
written for Rockbox. There are a wide variety of these supplied with
Rockbox, including several games, some impressive demos and a number of
utilities. A detailed description of the different plugins is to be found in
\reference{ref:plugins}.

\section{\label{ref:Info}System}

\begin{description}
\item[Rockbox Info:]
  Displays some basic system information. This is, from top to bottom,
  the amount of memory Rockbox has available for storing music (the buffer).
  The battery status.
\opt{multivolume}{%
  Memory size and amount of free space on the two data volumes, this info is
  given separately for internal memory (\emph{Int}) and for a plugged in
  memory card
  \opt{sansa,e200v2,fuze,fuzev2,clipplus,clipzip}{(\emph{MSD})}.
}%
\nopt{multivolume}{Hard disk size and the amount of free space on the disk.}

\item[Credits:]
  Display the list of contributors.

\item[Running Time:]
  Shows the runtime of your \dap{} in hours, minutes and seconds.
  \begin{description}
    \item[Running Time:]
        This item shows the cumulative overall runtime of your \dap{} since you 
        either disconnected it from charging (in Rockbox) or manually 
        reset this item. A manual reset is done through pressing any button, 
        followed by pressing \ActionStdOk{}.
    \item[Top Time:]
        This item shows the cumulative overall runtime of your \dap{} since you 
        last manually reset this item. A manual reset is done through pressing 
        any button, followed by pressing \ActionStdOk{}.
  \end{description}

\item[Debug (Keep Out!):]
  This sub menu is intended to be used \emph{only} by Rockbox developers.
  It shows hardware, disk, battery status and other technical information.  
  \warn{It is not recommended that users access this menu unless instructed to
  do so in the course of fixing a problem with Rockbox. If you think you have 
  messed up your settings by use of this menu please try to reset \emph{all} 
  settings before asking for help.}
\end{description}

\opt{quickscreen}
{
\section{\label{ref:QuickScreen}Quick Screen}
  \nopt{erosqnative}{
    Although the \setting{Quick Screen} is accessible from nearly everywhere,
    not just the \setting{Main Menu}, it is worth mentioning here.
  }
  \opt{erosqnative}{
    On the \playertype{}, The \setting{Quick Screen} is only
    available from the \setting{While Playing Screen} by default.
  }
  It allows rapid access to your four favourite settings.  The default settings are
  \setting{Shuffle} (\reference{ref:PlaybackSettings}) and
  \setting{Repeat} (\reference{ref:PlaybackSettings}), but almost all
  configurable options in Rockbox can be placed on this screen.  To change the
  options, navigate through the menus to the setting you want to add and press
  \ActionStdContext.  In the menu which appears you will be given options
  to place the setting on the \setting{Quick Screen}.
  
  Press
  \nopt{erosqnative}{\ActionStdQuickScreen{}}
  \opt{erosqnative}{\ActionWpsQuickScreen{}} to access it and \ActionQuickScreenExit{} to exit.
  The direction buttons will modify the individual setting values as indicated 
  by the arrow icons. Please note that the settings at opposite sides of the
   screen cycle through the available options in opposite directions.
   Therefore if you select the same setting at e.g. the top and bottom of the
   quickscreen, then pressing up and down will cycle through this setting in
   opposite directions.

  \opt{ipod}{
    \note{You can always adjust the sound volume on the QuickScreen.
    To do so, slide your finger around the click wheel as you would on the
    While Playing Screen.}
  }

  Press \ActionStdContext{} to access the \setting{Shortcuts Menu} directly from the
  \setting{Quick Screen}.
}

\section{\label{ref:MainMenuShortcuts}Shortcuts}

This menu item is a container for user defined shortcuts to files, folders or
settings. With a shortcut,
\begin{itemize}
  \item A file can be ``run'' (i.e. a music file played, plugin started or
        a \fname{.cfg} loaded)
  \item The file browser can be opened with the cursor positioned at
        a specified file or folder
  \item A file's or folder's ``Playing Next...'' context menu item can
        be displayed
  \item A settings current value (any which can be added to the
        \setting{Quick Screen})
  \item A setting can be configured (any which can be added to the
        \setting{Quick Screen})
  \item A debug menu item can be displayed (useful for developers mostly)
\opt{rtc}{
  \item The current time can be spoken
}
  \item The sleep timer can be configured
  \item The \dap{} can be turned off or rebooted
\end{itemize}

\note{Shortcuts into the database are not possible}

Shortcuts are loaded from the file \fname{/.rockbox/shortcuts.txt} which lists
each item to be displayed. Each shortcut looks like the following:

\begin{example}
    [shortcut]
    type: <shortcut type>
    data: <what the shortcut actually links to>
    name: <what the shortcut should be displayed as>
    icon: <number of the theme icon to use (see \wikilink{CustomIcons})>
    talkclip: <full path to a talk clip to speak when voice menus are enabled> (example: \fname{/.rockbox/filename.talk})
\end{example}

Only ``type'' and ``data'' are required (except if type is ``separator'' in which case
``data'' is also not required).

Available types are:
\begin{description}
\item[file] \config{data} is the name of the file to ``run''
\item[browse] \config{data} is the file or the folder to open the file browser at
\item[playlist menu] \config{data} is the file or the folder to open the
  ``Playing Next...'' context menu item on (see \reference{ref:playingnext_submenu})
\item[setting] \config{data} is the config name of the setting you want to change
  (see \reference{ref:config_file_options} for the list of the possible settings)
\item[apply] \config{data} is the config name of the setting followed by ':' 
  and the value you want applied. ex. 'data: volume: 20' would set the volume to 20 when selected
  (see \reference{ref:config_file_options} for the list of the possible settings)
\item[debug] \config{data} is the name of the debug menu item to display
\item[separator] \config{data} is ignored; \config{name} can be used to display text,
  or left blank to make the list more accessible with visual gaps
\item[time] \config{data} needs to be \opt{rtc}{either ``talk'' to talk the time, or }``sleep X''
  where X can, optionally, be the number of minutes to run the sleep timer for (0 to disable).
  If ``sleep'' is not followed by a number, the sleep timer can be stopped, if running,
  or started using the default duration; \config{name} will be ignored in that case. Otherwise
  \config{name} is required for this shortcut type.
\item[shutdown] \config{data} is ignored; \config{name} can be used to display text
\item[reboot] \config{data} is ignored; \config{name} can be used to display text
\end{description}

If the name/icon items are not specified, a sensible default will be used.

\note{For the ``browse'' type, if you want the file browser to start \emph{inside}
a folder, make sure the data has the trailing slash (i.e \fname{/Music/} instead of
\fname {/Music}). Without the trailing slash, it will cause the file browser to open
with \fname{/Music} selected instead.}

The file \fname{shortcuts.txt} can be edited with any text editor. Most items can
also be added to it through their context menu item ``Add to shortcuts''.
A reboot is needed for manual changes to \fname{shortcuts.txt} to be applied.

Shortcuts can be manually removed by selecting the one you wish to remove and pressing
\ActionStdContext{}.

\input{rockbox_interface/hotkeys.tex}

