% $Id$ %
\chapter{Advanced Topics}

\section{\label{ref:CustomisingUI}Customising the User Interface}

\subsection{\label{ref:CustomisingTheMainMenu}Customising The Main Menu}

The easiest way to reorder or to hide menu items is via the settings menu
(see \reference{ref:main_menu_config}).

To accomplish this using a text editor instead, load a \fname{.cfg} file
(as described in \reference{ref:manage_settings}) containing the following line:
\config{root~menu~order:items}, where ``items'' is a comma separated list
(no spaces around the commas!) of the following
words: \config{bookmarks}, \config{files}, \opt{tagcache}{\config{database}, }%
\config{wps}, \config{settings}, \opt{recording}{\config{recording}, }%
\opt{radio}{\config{radio}, }\config{playlists}, \config{plugins},
\config{system\_menu}, \config{shortcuts}.
Each of the words, if it occurs in the list, activates the appropriate item
in the main menu. The order of the items is given by the order of the words
in the list. The items whose words do not occur in the list will be hidden,
with one exception: the menu item \setting{Settings} will be shown even if
its word is not in the list (it is added as the last item then).

The following configuration example will change the main menu so that it will
contain only the items for the file browser, for resuming the playback, and
for changing the settings (the latter will be added automatically).
\begin{example}
    \config{root menu order:files,wps}
\end{example}


To reset the menu items to the default, use \config{root~menu~order:-} (i.e.
use a hyphen instead of ``items'').

\subsection{\label{ref:CustomisingThePlayername}Customising The Playername}

Some themes show a customizable playername in the Whats Playing Screen.
Edit the first line of \fname{/.rockbox/playername.txt} to show your own message
or leave an empty file to disable the feature. Deleting the file will generate a new
playername.txt file containing ``\playername'' on the next boot.

\subsection{\label{ref:OpenPlugins}Open Plugin Menu Items}

Rockbox allows you to choose a plugin to run for select menu options.
Simply choose the option in the setting menu and choose the plugin
you would like to run.

\subsection{\label{ref:GettingExtras}Getting Extras}

Rockbox supports custom fonts. A collection of fonts is available for download
in the font package at \url{https://www.rockbox.org/daily.shtml}.

\subsection{\label{ref:Loadingfonts}Loading Fonts}\index{Fonts}
Rockbox can load fonts dynamically. Simply copy the \fname{.fnt} file to the
\dap{} and ``play'' it in the \setting{File Browser}. If you want a font to
be loaded automatically every time you start up, it must be located in the
\fname{/.rockbox/fonts} directory and the filename must be at most 24 characters
long. You can browse the fonts in \fname{/.rockbox/fonts} under
\setting{Settings $\rightarrow$ Theme Settings $\rightarrow$ Font}
in the \setting{Main Menu}.\\

\note{Advanced Users Only: Any BDF font should
  be usable with Rockbox. To convert from \fname{.bdf} to \fname{.fnt}, use
  the \fname{convbdf} tool. This tool can be found in the \fname{tools}
  directory of the Rockbox source code. See \wikilink{CreateFonts\#ConvBdf}
  for more details. Or just run \fname{convbdf} without any parameters to
  see the possible options.}

\subsection{\label{ref:Loadinglanguages}Loading Languages}
\index{Language files}%
Rockbox can load language files at runtime. Simply copy the \fname{.lng} file
\emph{(do not use the .lang file)} to the \dap\ and ``play'' it in the
Rockbox directory browser or select \setting{Settings $\rightarrow$
General Settings $\rightarrow$ Language }from the \setting{Main Menu}.\\

\note{If you want a language to be loaded automatically every time you start
up, it must be located in the \fname{/.rockbox/langs} directory and the filename
must be a maximum of 24 characters long.\\}

If your language is not yet supported and you want to write your own language
file find the instructions on the Rockbox website:
\wikilink{LangFiles}

\opt{lcd_color}{
  \subsection{\label{ref:ChangingFiletypeColours}Changing Filetype Colours}
  Rockbox has the capability to modify the \setting{File Browser} to show
  files of different types in different colours, depending on the file extension.

  \subsubsection{Set-up}
  There are two steps to changing the filetype colours -- creating
  a file with the extension \fname{.colours} and then activating it using
  a config file.  The \fname{.colours} files \emph{must} be stored in
  the \fname{/.rockbox/themes/} directory.
  The \fname{.colours} file is just a text file, and can be edited with
  your text editor of choice.

  \subsubsection{Creating the .colours file}
  The \fname{.colours} file consists of the file extension
  (or \fname{folder}) followed by a colon and then the colour desired
  as an RGB value in hexadecimal, as in the following example:\\*
  \\
  \config{folder:808080}\\
  \config{mp3:00FF00}\\
  \config{ogg:00FF00}\\
  \config{txt:FF0000}\\
  \config{???:FFFFFF}\\*

  The permissible extensions are as follows:\\*
  \\
  \config{folder, m3u, m3u8, cfg, wps, lng, rock, bmark, cue, colours, mpa,
  \firmwareextension{}, %
  mp1, mp2, mp3, ogg, oga, wma, wmv, asf, wav, flac, ac3, a52, mpc, wv,
  m4a, m4b, mp4, mp4a, mod, shn, aif, aiff, spx, sid, adx, nsf, nsfe, spc,
  ape, mac, sap, mpg, mpeg%
  \opt{HAVE_REMOTE_LCD}{, rwps}%
  \opt{lcd_non-mono}{, bmp}%
  \opt{radio}{, fmr}%
  , fnt, kbd}\\*
  %It'd be ideal to get these from filetypes.c

  All file extensions that are not either specifically listed in the
  \fname{.colours} files or are not in the list above will be
  set to the colour given by \config{???}. Extensions that
  are in the above list but not in the \fname{.colours}
  file will be set to the foreground colour as normal.

  \subsubsection{Activating}
  To activate the filetype colours, the \fname{.colours} file needs to be
  invoked from a \fname{.cfg} configuration file. The easiest way to do
  this is to create a new text file containing the following single
  line:\\*
  \\
  \config{filetype colours: /.rockbox/themes/filename.colours}\\*

  where filename is replaced by the filename you used when creating the
  \fname{.colours} file. Save this file as e.g. \fname{colours.cfg} in the
  \fname{/.rockbox/themes} directory and then activate the config file
  from the menu as normal
  (\setting{Settings} $\rightarrow$ \setting{Theme Settings}%
  $\rightarrow$ \setting{Browse Theme Files}).

  \subsubsection{Editing}
  The built-in \setting{Text Editor} (see \reference{sec:text_editor})
  automatically understands the
  \fname{.colours} file format, but an external text editor can
  also be used. To edit the \fname{.colours} file using Rockbox,
  ``play'' it in the \setting{File Browser}. The file will open in
  the \setting{Text Editor}. Upon selecting a line, the following choices
  will appear:\\*
  \\
  \config{Extension}\\
  \config{Colour}\\*

  If \config{Extension} is selected, the \setting{virtual keyboard}
  (see \reference{sec:virtual_keyboard}) appears,
  allowing the file extension to be modified. If \config{Colour}
  is selected, the colour selector screen appears. Choose the desired
  colour, then save the \fname{.colours} file using the standard
  \setting{Text Editor} controls.
}

\opt{lcd_non-mono}{%
  \subsection{\label{ref:LoadingBackdrops}Loading Backdrops}
  Rockbox supports showing an image as a backdrop in the \setting{File Browser}
  and the menus. The backdrop image must be a \fname{.bmp} file of the exact
  same dimensions as the display in your \dap{} (\dapdisplaysize{} with the last
  number giving the colour depth in bits). To use an image as a backdrop browse
  to it in the \setting{File Browser} and open the \setting{Context Menu}
  (see \reference{ref:Contextmenu}) on it and select the option
  \setting{Set As Backdrop}. If you want rockbox to remember your
  backdrop the next time you start your \dap{} the backdrop must be placed in
  the \fname{/.rockbox/backdrops} directory.
}%

\subsection{UI Viewport}
By default, the UI is drawn on the whole screen. This can be changed so that
the UI is confined to a specific area of the screen, by use of a UI
viewport. This is done by adding the following line to the
\fname{.cfg} file for a theme:\\*

\nopt{lcd_non-mono}{\config{ui viewport: X,Y,[width],[height],[font]}}
\nopt{lcd_color}{\opt{lcd_non-mono}{
    \config{ui viewport: X,Y,[width],[height],[font],[fgshade],[bgshade]}}}
\opt{lcd_color}{
    \config{ui viewport: X,Y,[width],[height],[font],[fgcolour],[bgcolour]}}
\\*

\opt{HAVE_REMOTE_LCD}{
  The dimensions of the menu that is displayed on the remote control of your
  \dap\ can be set in the same way.  The line to be added to the theme
  \fname{.cfg} is the following:\\*

  \nopt{lcd_non-mono}{\config{remote ui viewport: X,Y,[width],[height],[font]}}
  \nopt{lcd_color}{\opt{lcd_non-mono}{
    \config{remote ui viewport: X,Y,[width],[height],[font],[fgshade],[bgshade]}}}
  \opt{lcd_color}{
    \config{remote ui viewport: X,Y,[width],[height],[font],[fgcolour],[bgcolour]}}
\\*

}

  Only the first two parameters \emph{have} to be specified, the others can
  be omitted using `$-$' as a placeholder. The syntax is very similar to WPS
  viewports (see \reference{ref:Viewports}).  Briefly:

  \nopt{lcd_non-mono}{\input{advanced_topics/viewports/mono-uivp-syntax.tex}}
  \nopt{lcd_color}{\opt{lcd_non-mono}{\input{advanced_topics/viewports/grayscale-uivp-syntax.tex}}}
  \opt{lcd_color}{\input{advanced_topics/viewports/colour-uivp-syntax.tex}}

\section{\label{ref:ConfiguringtheWPS}Configuring the Theme}

\subsection{Themeing -- General Info}

  There are various different aspects of the Rockbox interface
  that can be themed -- the WPS or \setting{While Playing Screen}, the FMS or
  \setting{FM Screen} (if the \dap{} has a tuner), and the SBS or
  \setting{Base Skin}. The WPS is the name used to
  describe the information displayed on the \daps{} screen whilst an audio
  track is being played, the FMS is the screen shown while listening to the
  radio, and the SBS lets you specify a base skin that is shown in the
  menus and browsers, as well as the WPS and FMS. The SBS also allows you to
  control certain aspects of the appearance of the menus/browsers.
  There are a number of themes included in Rockbox, and
  you can load one of these at any time by selecting it in
  \setting{Settings $\rightarrow$ Theme Settings $\rightarrow$ Browse Theme Files}.
  It is also possible to set individual items of a theme from within the
  \setting{Settings $\rightarrow$ Theme Settings} menu.


\subsection{\label{ref:CreateYourOwnWPS}Themes -- Create Your Own}
The theme files are simple text files, and can be created (or edited) in your
favourite text editor. To make sure non-English characters
display correctly in your theme you must save the theme files with UTF-8
character encoding. This can be done in most editors, for example Notepad in
Windows 2000 or XP (but not in 9x/ME) can do this.

\begin{description}
\item [Files Locations: ] Each different ``themeable'' aspect requires its own file --
  WPS files have the extension \fname{.wps}, FM screen files have the extension
  \fname{.fms}, and SBS files have the extension \fname{.sbs}. The main theme
  file has the extension \fname{.cfg}. All files should have the same name.

  The theme \fname{.cfg} file should be placed in the \fname{/.rockbox/themes}
  directory, while the \fname{.wps}, \fname{.fms} and \fname{.sbs} files should
  be placed in the \fname{/.rockbox/wps} directory. Any images used by the
  theme should be placed in a subdirectory of \fname{/.rockbox/wps} with the
  same name as the theme, e.g. if the theme files are named
  \fname{mytheme.wps, mytheme.sbs} etc., then the images should be placed in
  \fname{/.rockbox/wps/mytheme}.
\end{description}

All full list of the available tags are given in appendix
\reference{ref:wps_tags}; some of the more powerful concepts in theme design
are discussed below.

\begin{itemize}
\item All characters not preceded by \% are displayed as typed.
\item Lines beginning with \# are comments and will be ignored.
\end{itemize}

\note{Keep in mind that your \daps{} resolution is \dapdisplaysize{} (with
  the last number giving the colour depth in bits) when
  designing your own WPS, or if you use a WPS designed for another target.
  \opt{HAVE_REMOTE_LCD}{The resolution of the remote is
      \opt{iriverh100,iriverh300}{128$\times$64$\times$1}%
      \opt{iaudiox5,iaudiom5,iaudiom3}{128$\times$96$\times$2}
      pixels.
  }
}

\subsubsection{\label{ref:Viewports}Viewports}

By default, a viewport filling the whole screen contains all the elements
defined in each theme file. The
\opt{lcd_non-mono}{elements in this viewport are displayed
  with the same background/\linebreak{}foreground
  \opt{lcd_color}{colours}\nopt{lcd_color}{shades} and the}
text is rendered in the
same font as in the main menu. To change this behaviour a custom viewport can
be defined. A viewport is a rectangular window on the screen%
\opt{lcd_non-mono}{ with its own foreground/background
\opt{lcd_color}{colours}\nopt{lcd_color}{shades}}.
This window also has variable dimensions. To
define a viewport a line starting \config{{\%V(\dots)}} has to be
present in the theme file. The full syntax will be explained later in
this section. All elements placed before the
line defining a viewport are displayed in the default viewport. Elements
defined after a viewport declaration are drawn within that viewport.
Loading images (see Appendix \reference{ref:wps_images})
should be done within the default viewport.
A viewport ends either with the end of the file, or with the next viewport
declaration line. Viewports sharing the same
coordinates and dimensions cannot be displayed at the same time. Viewports
cannot be layered \emph{transparently} over one another. Subsequent viewports
will be drawn over any other viewports already drawn onto that
area of the screen.


\nopt{lcd_non-mono}{\input{advanced_topics/viewports/mono-vp-syntax.tex}}
\nopt{lcd_color}{\opt{lcd_non-mono}{\input{advanced_topics/viewports/grayscale-vp-syntax.tex}}}
\opt{lcd_color}{\input{advanced_topics/viewports/colour-vp-syntax.tex}}

\opt{lcd_non-mono}{
\subsubsection{Viewport Line Text Styles}
  \begin{tagmap}
    \config{\%Vs(mode[,param])}
    & Set the viewport text style to `mode' from this point forward\\
  \end{tagmap}

Mode can be the following:

\begin{rbtabular}{.75\textwidth}{lX}{\textbf{Mode} & \textbf{Description}}{}{}
  clear & Restore the default style\\
  invert & Draw lines inverted\\
  color & Draw the text coloured by the value given in `param'. Functionally
    equivalent to using the \%Vf() tag\\
  \opt{lcd_color}{%
    gradient & Draw the next `param' lines using a gradient as
    defined by \%Vg. By default the gradient is drawn over 1 line.
    \%Vs(gradient,2) will use 2 lines to fully change from the start colour to
    the end colour\\}
\end{rbtabular}
}
\subsubsection{Conditional Viewports}

Any viewport can be displayed either permanently or conditionally.
Defining a viewport as \config{{\%V(\dots)}}
will display it permanently.

\begin{itemize}
\item {\config{\%Vl('identifier',\dots)}}
This tag preloads a viewport for later display. `identifier' is a single
lowercase letter (a-z) and the `\dots' parameters use the same logic as
the \config{\%V} tag explained above.
\item {\config{\%Vd('identifier')}} Display the `identifier' viewport.
\end{itemize}

Viewports can share identifiers so that you can display multiple viewports
with one \%Vd line.

\nopt{lcd_non-mono}{\input{advanced_topics/viewports/mono-conditional.tex}}
\nopt{lcd_color}{%
  \opt{lcd_non-mono}{\input{advanced_topics/viewports/grayscale-conditional.tex}}}
\opt{lcd_color}{\input{advanced_topics/viewports/colour-conditional.tex}}
\\*

\note{The tag to display conditional viewports must come before the tag to
preload the viewport in the \fname{.wps} file.}

\subsection{Info Viewport (SBS only)}
As mentioned above, it is possible to set a UI viewport via the theme
\fname{.cfg} file. It is also possible to set the UI viewport through the SBS
file, and to conditionally select different UI viewports.

  \begin{itemize}
    \item {\config{\%Vi('label',\dots)}}
    This viewport is used as Custom UI Viewport in the case that the theme
    doesn't have a ui viewport set in the theme \fname{.cfg} file. Having this
    is strongly recommended since it makes you able to use the SBS
    with other themes. If label is set this viewport can be selectivly used as the
    Info Viewport using the \%VI tag. The `\dots' parameters use the same logic as
    the \config{\%V} tag explained above.

    \item {\config{\%VI('label')}} Set the Info Viewport to use the viewport called
    label, as declared with the previous tag.
  \end{itemize}

\subsection{\label{ref:multifont}Additional Fonts}
Additional fonts can be loaded within each screen file to be used in that
screen. In this way not only can you have different fonts between e.g. the menu
and the WPS, but you can use multiple fonts in each of the individual screens.\\

\config{\%Fl('id',filename,glyphs)}

  \begin{itemize}
    \item `id' is the number you want to use in viewport declarations, 0 and 1
       are reserved and so can't be used.
    \item `filename' is the font filename to load. Fonts should be stored in
       \fname{/.rockbox/fonts/}
    \item `glyphs' is an optional specification of how many unique glyphs to
       store in memory. Default is from the system setting
       \setting{Glyphs To Load}.
  \end{itemize}

  An example would be: \config{\%Fl(2,12-Nimbus.fnt,100)}

\subsubsection{Conditional Tags}

\begin{description}
\item[If/else: ]
Syntax: \config{\%?xx{\textless}true{\textbar}false{\textgreater}}

If the tag specified by ``\config{xx}'' has a value, the text between the
``\config{{\textless}}'' and the ``\config{{\textbar}}'' is displayed (the true
part), else the text between the ``\config{{\textbar}}'' and the
``\config{{\textgreater}}'' is displayed (the false part).
The else part is optional, so the ``\config{{\textbar}}'' does not have to be
specified if no else part is desired. The conditionals nest, so the text in the
if and else part can contain all \config{\%} commands, including conditionals.

\item[Enumerations: ]
Syntax: \config{\%?xx{\textless}alt1{\textbar}alt2{\textbar}alt3{\textbar}\dots{\textbar}else{\textgreater}}

For tags with multiple values, like Play status, the conditional can hold a
list of alternatives, one for each value the tag can have.
Example enumeration:
\begin{example}
     \%?mp{\textless}Stop{\textbar}Play{\textbar}Pause{\textbar}Ffwd{\textbar}Rew{\textgreater}
\end{example}

The last else part is optional, and will be displayed if the tag has no value.
The WPS parser will always display the last part if the tag has no value, or if
the list of alternatives is too short.
\end{description}

\subsubsection{Next Song Info}
You can display information about the next song -- the song that is
about to play after the one currently playing (unless you change the
plan).

If you use the upper-case versions of the
three tags: \config{F}, \config{I} and \config{D}, they will instead refer to
the next song instead of the current one. Example: \config{\%Ig} is the genre
name used in the next song and \config{\%Ff} is the mp3 frequency.\\

\note{The next song information \emph{will not} be available at all
  times, but will most likely be available at the end of a song. We
  suggest you use the conditional display tag a lot when displaying
  information about the next song!}

\subsubsection{\label{ref:AlternatingSublines}Alternating Sublines}

It is possible to group items on each line into 2 or more groups or
``sublines''. Each subline will be displayed in succession on the line for a
specified time, alternating continuously through each defined subline.

Items on a line are broken into sublines with the semicolon
`\config{;}' character. The display time for
each subline defaults to 2 seconds unless modified by using the
`\config{\%t}' tag to specify an alternate
time (in seconds and optional tenths of a second) for the subline to be
displayed.

Subline related special characters and tags:
\begin{description}
\item[;] Split items on a line into separate sublines
\item[\%t] Set the subline display time. The
`\config{\%t}' is followed by either integer seconds (\config{\%t5[,timehide]}), or seconds
and tenths of a second within () e.g. (\config{\%t(3.5)[,timehide]}).
specify timehide to set number of seconds before tag will be displayed again.
when using timehide the message is displayed for timeshow and not displayed again for timehide seconds.
``Note:'' the max timeout allowed in themes is about 655 seconds (~10 minutes) going over will result in a timeout less than 10 minutes.
        \%t(ts,th) supports longer timeouts (~100 minutes)
\end{description}

Each alternating subline can still be optionally scrolled while it is
being displayed, and scrollable formats can be displayed on the same
line with non{}-scrollable formats (such as track elapsed time) as long
as they are separated into different sublines.
Example subline definition:
\begin{example}
     %s%t(4)%ia;%s%it;%t(3)%pc %pr : Display id3 artist for 4 seconds,
                                 Display id3 title for 2 seconds,
                                 Display current and remaining track time
                                 for 3 seconds,
                                 repeat...

     %s%t(4, 30)%Ia;%s%It;%t(3)%sId; : Display next id3 artist for 4 seconds,
                                 Display next id3 title for 2 seconds,
                                 Display next id3 album for 3 seconds
                                 For the next 30 seconds - display title for 2 seconds and album for 3 seconds
                                 repeat...
\end{example}

Conditionals can be used with sublines to display a different set and/or number
of sublines on the line depending on the evaluation of the conditional.
Example subline with conditionals:
\begin{example}
    %?it{\textless}%t(8)%s%it{\textbar}%s%fn{\textgreater};%?ia{\textless}%t(3)%s%ia{\textbar}%t(0){\textgreater}\\
\end{example}

The format above will do two different things depending if ID3 tags are
present. If the ID3 artist and title are present:
\begin{itemize}
\item Display id3 title for 8 seconds,
\item Display id3 artist for 3 seconds,
\item repeat\dots
\end{itemize}
If the ID3 artist and title are not present:
\begin{itemize}
\item Display the filename continuously.
\end{itemize}
Note that by using a subline display time of 0 in one branch of a conditional,
a subline can be skipped (not displayed) when that condition is met.

\subsubsection{Using Images}
You can have as many as 52 images in your WPS. There are various ways of
displaying images:
\begin{enumerate}
  \item Load and always show the image, using the \config{\%x} tag
  \item Preload the image with \config{\%xl} and show it with \config{\%xd}.
    This way you can have your images displayed conditionally.
  \item Load an image and show as backdrop using the \config{\%X} tag. The
    image must be of the same exact dimensions as your display.
\end{enumerate}

\optv{swcodec}{% This doesn't depend on swcodec but we don't have a \noptv
               % command.
  Example on background image use:
  \begin{example}
    %X(background.bmp)
  \end{example}
  The image with filename \fname{background.bmp} is loaded and used in the WPS.
}%

Example on bitmap preloading and use:
\begin{example}
    %x(a,static_icon.bmp,50,50)
    %xl(b,rep\_off.bmp,16,64)
    %xl(c,rep\_all.bmp,16,64)
    %xl(d,rep\_one.bmp,16,64)
    %xl(e,rep\_shuffle.bmp,16,64)
    %?mm<%xd(b)|%xd(c)|%xd(d)|%xd(e)>
\end{example}
Four images at the same x and y position are preloaded in the example. Which
image to display is determined by the \config{\%mm} tag (the repeat mode).

\subsubsection{Example File}
\begin{example}
    %s%?in<%in - >%?it<%it|%fn> %?ia<[%ia%?id<, %id>]>
    %pb%pc/%pt
\end{example}
That is, ``tracknum -- title [artist, album]'', where most fields are only
displayed if available. Could also be rendered as ``filename'' or ``tracknum --
title [artist]''.

%\begin{verbatim}
%  %s%?it<%?in<%in. |>%it|%fn>
%  %s%?ia<%ia|%?d2<%d(2)|(root)>>
%  %s%?id<%id|%?d1<%d(1)|(root)>> %?iy<(%iy)|>
%
%  %al%pc/%pt%ar[%pp:%pe]
%  %fbkBit %?fv<avg|> %?iv<(id3v%iv)|(no id3)>
%  %pb
%  %pm
%
%\end{verbatim}

\section{\label{ref:manage_settings}Managing Rockbox Settings}

\subsection{Introduction to \fname{.cfg} Files}
Rockbox allows users to store and load multiple settings through the use of
configuration files. A configuration file is simply a text file with the
extension \fname{.cfg}.

A configuration file may reside anywhere on the disk. Multiple
configuration files are permitted. So, for example, you could have
a \fname{car.cfg} file for the settings that you use while playing your
jukebox in your car, and a \fname{headphones.cfg} file to store the
settings that you use while listening to your \dap{} through headphones.

See \reference{ref:cfg_specs} below for an explanation of the format
for configuration files. See \reference{ref:manage_settings_menu} for an
explanation of how to create, edit and load configuration files.

\subsection{\label{ref:cfg_specs}Specifications for \fname{.cfg} Files}

The Rockbox configuration file is a plain text file, so once you use the
\setting{Save .cfg file} option to create the file, you can edit the file on
your computer using any text editor program. See
Appendix \reference{ref:config_file_options} for available settings. Configuration
files use the following formatting rules: %

\begin{enumerate}
\item Each setting must be on a separate line.
\item Each line has the format ``setting: value''.
\item Values must be within the ranges specified in this manual for each
  setting.
\item Lines starting with \# are ignored. This lets you write comments into
  your configuration files.
\end{enumerate}

Example of a configuration file:
\begin{example}
    volume: 70
    bass: 11
    treble: 12
    balance: 0
    time format: 12hour
    volume display: numeric
    show files: supported
    wps: /.rockbox/car.wps
    lang: /.rockbox/afrikaans.lng
\end{example}

\note{As you can see from the example, configuration files do not need to
  contain all of the Rockbox options.  You can create configuration files
  that change only certain settings. So, for example, suppose you
  typically use the \dap{} at one volume in the car, and another when using
  headphones. Further, suppose you like to use an inverse LCD when you are
  in the car, and a regular LCD setting when you are using headphones. You
  could create configuration files that control only the volume and LCD
  settings. Create a few different files with different settings, give
  each file a different name (such as \fname{car.cfg},
  \fname{headphones.cfg}, etc.), and you can then use the \setting{Browse .cfg
    files} option to quickly change settings.\\}

  A special case configuration file can be used to force a particular setting
  or settings every time Rockbox starts up (e.g. to set the volume to a safe
  level). Format a new configuration file as above with the required setting(s)
  and save it into the \fname{/.rockbox} directory with the filename
  \fname{fixed.cfg}.

\subsection{\label{ref:manage_settings_menu}The \setting{Manage Settings}
  menu} The \setting{Manage Settings} menu can be found in the \setting{Main
  Menu}. The \setting{Manage Settings} menu allows you to save and load
  \fname{.cfg} files.

\begin{description}

\item [Browse .cfg Files]Opens the \setting{File Browser} in the
  \fname{/.rockbox} directory and displays all \fname{.cfg} (configuration)
  files. Selecting a \fname{.cfg} file will cause Rockbox to load the settings
  contained in that file. Pressing \ActionStdCancel{} will exit back to the
  \setting{Manage Settings} menu. See the \setting{Write .cfg files} option on
  the \setting{Manage Settings} menu for details of how to save and edit a
  configuration file.

\item [Reset Settings]This wipes the saved settings
  in the \dap{} and resets all settings to their default values.

  \opt{IRIVER_H100_PAD,IRIVER_H300_PAD,IAUDIO_X5_PAD,SANSA_E200_PAD,SANSA_C200_PAD%
      ,PBELL_VIBE500_PAD,SAMSUNG_YH92X_PAD,SAMSUNG_YH820_PAD}{
      \note{You can also reset all settings to their default
      values by turning off the \dap, turning it back on, and holding the
      \ButtonRec{} button immediately after the \dap{} turns on.}
  }
  \opt{IRIVER_H10_PAD}{\note{You can also reset all settings to
      their default values by turning off the \dap, and turning it back on
      with the \ButtonHold{} button on.}
  }
  \opt{IPOD_4G_PAD}{\note{You can also reset all settings to their default
      values by turning off the \dap, turning it back on, and activating the
      \ButtonHold{} button immediately after the backlight comes on.}
  }
  \opt{GIGABEAT_PAD}{\note{You can also reset all settings to their default
      values by turning off the \dap, turning it back on and pressing the
      \ButtonA{} button immediately after the \dap{} turns on.}
  }

\item [Save .cfg File]This option writes a \fname{.cfg} file to
  your \daps{} disk. The configuration file has the \fname{.cfg}
  extension and is used to store all of the user settings that are described
  throughout this manual.

  Hint: Use the \setting{Save .cfg File} feature (\setting{Main Menu
    $\rightarrow$ Manage Settings}) to save the current settings, then
  use a text editor to customize the settings file. See Appendix
  \reference{ref:config_file_options} for the full reference of available
  options.

\item [Save Sound Settings]This option writes a \fname{.cfg} file to
  your \daps{} disk. The configuration file has the \fname{.cfg}
  extension and is used to store all of the sound related settings.

\item [Save Theme Settings]This option writes a \fname{.cfg} file to
  your \daps{} disk. The configuration file has the \fname{.cfg}
  extension and is used to store all of the theme related settings.

\end{description}

\nopt{rgnano}{
\section{\label{ref:FirmwareLoading}Firmware Loading}

\subsection{\label{ref:using_rolo}Using ROLO (Rockbox Loader)}
Rockbox is able to load and start another firmware file without rebooting.
You just ``play'' a file with the extension %
\opt{iriverh100,iriverh300}{\fname{.iriver}.} %
\opt{ipod}{\fname{.ipod}.} %
\opt{iaudio}{\fname{.iaudio}.} %
\opt{sansa,iriverh10,iriverh10_5gb,mrobe100,vibe500,samsungyh}{\fname{.mi4}.} %
\opt{sansaAMS,fuzeplus}{\fname{.sansa}.} %
\opt{gigabeatf,gigabeats}{\fname{.gigabeat}.} %
\opt{fiiom3k}{\fname{.m3k}} %
\opt{shanlingq1}{\fname{.q1}} %
\opt{agptekrocker}{\fname{.rocker}} %
\opt{xduoox3ii}{\fname{.x3ii}} %
\opt{xduoox20}{\fname{.x20}} %
\opt{aigoerosq}{\fname{.erosq}} %
This can be used to test new firmware versions without deleting your
current version.
}

\opt{multi_boot}{
\subsection{\label{ref:using_multiboot}Using Multiboot}
  \newcommand{\redirectext}{<playername>}
  \opt{fuze}{\renewcommand*{\redirectext}{fuze}}
  \opt{fuzev2}{\renewcommand*{\redirectext}{fuze2}}
  \opt{fuzeplus}{\renewcommand*{\redirectext}{fuze+}}
  \opt{clipplus}{\renewcommand*{\redirectext}{clip+}}
  \opt{clipzip}{\renewcommand*{\redirectext}{clipzip}}
  \opt{e200}{\renewcommand*{\redirectext}{e200}}

  Using a specially crafted redirect file combined with special firmware builds
  that allow the boot drive to be passed from the bootloader, some devices can
  run Rockbox from a MicroSD card. You can even direct the bootloader to run the
  firmware from a different folder within the SD card if desired.
  \note{
    Each supported model looks for a specific redirect file
    (file extension is unique to each model).
  }

  Create \fname{/rockbox\_main.\redirectext{}} in the root of the SD card.
  The redirect file should contain a single line denoting the path to the root
  of the desired \fname{.rockbox} directory. A single forward slash ``/'' indicates
  \fname{/.rockbox} in the root of the sd card.
  \note{The redirect file does not work when placed on the internal drive.}

  If instead you wanted to run the Rockbox from SD card \fname{/mybuild/.rockbox}
  then your \fname{/rockbox\_main.\redirectext{}} file should contain:
  ``/mybuild/''
}

\section{Optimising battery runtime}
  Rockbox offers a lot of settings that have high impact on the battery runtime
  of your \dap{}. The largest power savings can be achieved through disabling
  unneeded hardware components -- for some of those there are settings
  available.


  Another area of savings is avoiding or reducing CPU boosting
  through disabling computing intense features (e.g. sound processing) or
  using effective audio codecs.
  The following provides a short overview of the most relevant settings and
  rules of thumb.

\subsection{Display backlight}
  The active backlight consumes a lot of power. Therefore choose a setting that
  disables the backlight after timeout (for setting \setting{Backlight} see
  \reference{ref:Displayoptions}). Avoid having the backlight enabled all the
  time (Activating \setting{selectivebacklight}
  \reference{ref:Displayoptions} can further reduce power consumption).

\opt{lcd_sleep}{
\subsection{Display power-off}
  Shutting down the display and the display controller saves a reasonable amount
  of power. Choose a setting that will put the display to sleep after timeout
  (for setting \setting{Sleep} see \reference{ref:Displayoptions}). Avoid to
  have the display enabled all the time -- even, if the display is transflective
  and is readable without backlight. Depending on your \dap{} it might be
  significantly more efficient to re-enable the display and its backlight for a
  glimpse a few times per hour than to keep the display enabled.
}

\opt{accessory_supply}{
\subsection{Accessory power supply}
  As default your \dap{}'s accessory power supply is always enabled to ensure
  proper function of connected accessory devices. Disable this power supply, if
  -- or as long as -- you do not use any accessory device with your \dap{} while
  running Rockbox (see \reference{ref:AccessoryPowerSupply}).
}

\opt{lineout_poweroff}{
\subsection{Line Out}
  Rockbox allows to switch off the line-out on your \dap{}. If you do not need
  the line-out, switch it off (see \reference{ref:LineoutOnOff}).
}

\opt{spdif_power}{
\subsection{Optical Output}
  Rockbox allows to switch off the S/PDIF output on your \dap{}. If you do not
  need this output, switch it off (see \reference{ref:SPDIF_OnOff}).
}

\opt{disk_storage}{
\subsection{Anti-Skip Buffer}
  Having a large anti-skip buffer tends to use more power, and may reduce your
  battery life. It is recommended to always use the lowest possible setting
  that allows correct and continuous playback (see \reference{ref:AntiSkipBuf}).
}

\subsection{Replaygain}
  Replaygain is a post processing that equalises the playback volume of audio
  files to the same perceived loudness. This post processing applies a factor
  to each single PCM sample and is therefore consuming additional CPU time. If
  you want to achieve some (minor) savings in runtime, switch this feature off
  (see \reference{ref:ReplayGain}).

\subsection{Peak Meter}
  The peak meter is a feature of the While Playing Screen and will be updated with a
  high framerate. Depending on your \dap{} this might result in a high CPU load. To
  save battery runtime you should switch this feature off (see \reference{ref:peak_meter}).
  \opt{ipodvideo}{
    \note{Especially the \playerman{} \playertype{} suffers from an enabled peak meter.}
  }

\subsection{Audio format and bitrate}
  In general the fastest decoding audio format will be the best in terms of
  battery runtime on your \dap{}. An overview of different codec's performance
  on different \dap{}s can be found at \wikilink{CodecPerformanceComparison}.

\opt{flash_storage}{
  Your target uses flash that consumes a certain amount of power during access.
  The less often the flash needs to be switched on for buffering and the shorter
  the buffering duration is, the lower is the overall power consumption.
  Therefore the bitrate of the audio files does have an impact on the battery
  runtime as well. Lower bitrate audio files will result in longer battery
  runtime.
}
\opt{disk_storage}{
  Your target uses a hard disk which consumes a large amount of power while
  spinning -- up to several hundred mA. The less often the hard disk needs to
  spin up for buffering and the shorter the buffering duration is, the lower is
  the power consumption. Therefore the bitrate of the audio files does have an
  impact on the battery runtime as well. Lower bitrate audio files will result
  in longer battery runtime.
}

  Please do not re-encode any existing audio files from one lossy format to
  another based upon the above mentioned. This will reduce the audio quality.
  If you have the choice, select the best suiting codec when encoding the
  original source material.

\subsection{Sound settings}
  In general all kinds of sound processing will need more CPU time and therefore
  consume more power. The less sound processing you use, the better it is for
  the battery runtime (for options see \reference{ref:configure_rockbox_sound}).
