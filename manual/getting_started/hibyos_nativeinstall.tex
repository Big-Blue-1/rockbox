We will install the bootloader with the original firmware's recovery
mode. The process is summed up as the following:

\begin{itemize}
  \item Determine what hardware version your player is and download
    the correct bootloader update file
  \item Place the bootloader \fname{update.upt} file on the SD card
  \item In the original firmware, run the Firmware Update: \\
    \fname{System Settings --> Firmware Update}
\end{itemize}

\subsubsection{Determine hardware version}\label{ref:determine_hardware_version}
Determine what hardware version your player is. Go to \fname{System Settings --> About The Player --> Version} and reference
the list below. hw1, hw1.5, and hw2 players all use the same update
file (with one exception), while hw3 players use a different one.

\note{Important: If your player's version is not contained in this list,
for example if the firmware version is newer than what is listed here,
we cannot be sure that the hardware is the same. The best thing to do is
contact the manufacturer and ask them two things: (1) for an update file
of your version, and (2) if a player with the most recent version listed
here can be upgraded to the firmware version on your player. If they say
yes, we can be more certain that the hardware has not changed. These lists
may not be the most up to date, please see the wiki page at
\url{https://www.rockbox.org/wiki/AIGOErosQK} for the most up-to-date list.}

\begin{description}
\item[hw1/hw1.5/hw2 players]
  \begin{itemize}
    \item Aigo Eros Q V1.8 - V2.0
    \item Hifiwalker H2 V1.1 - V1.6
    \item Surfans F20 V2.2 - V2.7
  \end{itemize}
  These players use \fname{erosqnative-hw1hw2-erosq.upt} as the update file.
  The lone exception is the Hifiwalker H2 V1.3, which uses the update file
  \fname{erosqnative-hw1hw2-eros\_h2.upt}.
\item[hw3 players]
  \begin{itemize}
    \item Aigo Eros Q V2.1
    \item Hifiwalker H2 V1.7 - V1.8
    \item Surfans F20 V3.0 - V3.3
  \end{itemize}
  These players use \fname{erosqnative-hw3-erosq.upt} as the update file.
\item[hw4 players]
  \begin{itemize}
    \item Aigo Eros Q V2.2
    \item Hifiwalker H2 V1.9 - V2.0
    \item Surfans F20 V3.4
  \end{itemize}
  These players use \fname{erosqnative-hw4-erosq_2024.upt} as the update file.
\end{description}

Download the \fname{.upt} file for these players from \download{bootloader/aigo/native/}.

\note{All players use the same Rockbox build, only the bootloader is different.}

\subsubsection{Place update file on SD card}\label{ref:place_on_sd_card}
Place the appropriate bootloader file on the root of the SD card and name it
\emph{exactly} \fname{update.upt}.

\note{This is a good time to ensure that your Rockbox installation \fname{.rockbox}
is present on your SD card.}

Don't forget to safely eject/unmount your player.

\subsubsection{Run Firmware Update}\label{ref:run_firmware_update}
In the original firmware, run the firmware updater by going to
\fname{System Settings --> Firmware Update}. At this point, you can delete
\fname{update.upt} from the SD card if you wish. \emph{Do not delete .rockbox,
this is your Rockbox installation and needs to stay there!}
